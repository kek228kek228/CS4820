\documentclass[12pt]{article}
%\usepackage{fullpage}
\usepackage{epic}
\usepackage{eepic}
\usepackage{paralist}
\usepackage{graphicx}
\usepackage{algorithm,algorithmic}
\usepackage{tikz}
\usepackage{xcolor,colortbl}
\usepackage{wrapfig}
\usepackage{amsmath}


%%%%%%%%%%%%%%%%%%%%%%%%%%%%%%%%%%%%%%%%%%%%%%%%%%%%%%%%%%%%%%%%
% This is FULLPAGE.STY by H.Partl, Version 2 as of 15 Dec 1988.
% Document Style Option to fill the paper just like Plain TeX.

\typeout{Style Option FULLPAGE Version 2 as of 15 Dec 1988}

\topmargin 0pt
\advance \topmargin by -\headheight
\advance \topmargin by -\headsep

\textheight 8.9in

\oddsidemargin 0pt
\evensidemargin \oddsidemargin
\marginparwidth 0.5in

\textwidth 6.5in
%%%%%%%%%%%%%%%%%%%%%%%%%%%%%%%%%%%%%%%%%%%%%%%%%%%%%%%%%%%%%%%%

\pagestyle{empty}
\setlength{\oddsidemargin}{0in}
\setlength{\topmargin}{-0.8in}
\setlength{\textwidth}{6.8in}
\setlength{\textheight}{9.5in}


\def\ind{\hspace*{0.3in}}
\def\gap{0.1in}
\def\bigap{0.25in}
\newcommand{\Xomit}[1]{}
\newcommand{\Mod}[1]{\ (\mathrm{mod}\ #1)}


\begin{document}

\setlength{\parindent}{0in}
\addtolength{\parskip}{0.1cm}
\setlength{\fboxrule}{.5mm}\setlength{\fboxsep}{1.2mm}
\newlength{\boxlength}\setlength{\boxlength}{\textwidth}
\addtolength{\boxlength}{-4mm}
\begin{center}\framebox{\parbox{\boxlength}{{\bf
CS 4820, Spring 2019 \hfill Homework 8, Problem 2}\\
% TODO: fill in your own name, netID, and collaborators
Name: \\
NetID: \\
}}
\end{center}
\vspace{5mm}
\vskip \bigap
{\bf (2)} {\em (10 points)} Consider the problem of NDPATH defined in the following way: An input instance is of the form $(G,S,T)$, where $G$ is an undirected graph on $n$ vertices,  $S=\{s_1,\ldots,s_k \}, T=\{t_1,\ldots,t_k\}$ are disjoint subsets of nodes of equal cardinality $k$ (for any integer $k \in \{1,\ldots,\lfloor n/2 \rfloor \})$. It is an `Yes' instance of NDPATH if there are node disjoint paths from $s_i$ to $t_i$, for all $i=1,\ldots,k$, and is a `NO' instance otherwise. Prove that NDPATH is NP-complete.  

\emph{Hint: Reduce from 3SAT. For each clause and each variable, add a pair of $s_i$'s and $t_i$'s. For each clause, introduce a few intermediate nodes, corresponding to the variables appearing in the clause.}


%% Your solution goes here.


\end{document}
