\documentclass[12pt]{article}
%\usepackage{fullpage}
\usepackage{epic}
\usepackage{eepic}
\usepackage{paralist}
\usepackage{graphicx}
\usepackage{algorithm,algorithmic}
\usepackage{tikz}
\usepackage{xcolor,colortbl}
\usepackage{wrapfig}
\usepackage{amsmath}


%%%%%%%%%%%%%%%%%%%%%%%%%%%%%%%%%%%%%%%%%%%%%%%%%%%%%%%%%%%%%%%%
% This is FULLPAGE.STY by H.Partl, Version 2 as of 15 Dec 1988.
% Document Style Option to fill the paper just like Plain TeX.

\typeout{Style Option FULLPAGE Version 2 as of 15 Dec 1988}

\topmargin 0pt
\advance \topmargin by -\headheight
\advance \topmargin by -\headsep

\textheight 8.9in

\oddsidemargin 0pt
\evensidemargin \oddsidemargin
\marginparwidth 0.5in

\textwidth 6.5in
%%%%%%%%%%%%%%%%%%%%%%%%%%%%%%%%%%%%%%%%%%%%%%%%%%%%%%%%%%%%%%%%

\pagestyle{empty}
\setlength{\oddsidemargin}{0in}
\setlength{\topmargin}{-0.8in}
\setlength{\textwidth}{6.8in}
\setlength{\textheight}{9.5in}


\def\ind{\hspace*{0.3in}}
\def\gap{0.1in}
\def\bigap{0.25in}
\newcommand{\Xomit}[1]{}
\newcommand{\Mod}[1]{\ (\mathrm{mod}\ #1)}


\begin{document}

\setlength{\parindent}{0in}
\addtolength{\parskip}{0.1cm}
\setlength{\fboxrule}{.5mm}\setlength{\fboxsep}{1.2mm}
\newlength{\boxlength}\setlength{\boxlength}{\textwidth}
\addtolength{\boxlength}{-4mm}
\begin{center}\framebox{\parbox{\boxlength}{{\bf
CS 4820, Spring 2019 \hfill Homework 4, Problem 1}\\
% TODO: fill in your own name, netID, and collaborators
Name: \\
NetID: \\
}}
\end{center}
\vspace{5mm}
\vskip \bigap
{\bf (1)}  \textbf{Parentheses}  {\em (10 points)}\newline You are helping out a sloppy friend with a math problem. He has written an expression of the form $a_1 O_1 a_2 O_2 \ldots O_{n-1} a_n$, where each $a_i$ is either FALSE denoted by $0$ or TRUE denoted by $1$, and each $O_i$ is the logical OR denoted by $\lor$ or  the logical XOR denoted by $\oplus$.  Unfortunately, he has forgotten to insert in parentheses in the expression which leads you to think of the following problem: how many ways of parenthesizing the expression leads the evaluation to be $0$? Develop an efficient algorithm that takes as input an expression $E$ of the form  $a_1 O_1 a_2 O_2 \ldots O_{n-1} a_n$  and outputs the number of ways of parenthesizing $E$ such that it evaluates to $0$ (i.e, FALSE).

Example: Let the input expression $E$ be: $0 \lor 1 \oplus 1 \lor 1$. Below are all the ways of parenthesizing $E$:
\begin{itemize}
 \item $(((0 \lor 1) \oplus 1) \lor 1)$ which evaluates to $1$.
  \item $((0 \lor (1 \oplus 1)) \lor 1)$ which evaluates to $1$.
 \item  $((0 \lor 1) \oplus (1 \lor 1))$which evaluates to $0$.
 \item $(0 \lor ((1 \oplus 1) \lor 1))$ which evaluates to $1$.
 \item $(0 \lor (1 \oplus (1 \lor 1)))$ which evaluates to $0$.
 \end{itemize}
Thus the output of your algorithm on this instance should be $2$.

\vskip \bigap

%% Your solution goes here.


\end{document}
