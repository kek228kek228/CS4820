\documentclass[12pt]{article}
%\usepackage{fullpage}
\usepackage{epic}
\usepackage{eepic}
\usepackage{paralist}
\usepackage{graphicx}
\usepackage{algorithm,algorithmic}
\usepackage{tikz}
\usepackage{xcolor,colortbl}
\usepackage{wrapfig}
\usepackage{amsmath}


%%%%%%%%%%%%%%%%%%%%%%%%%%%%%%%%%%%%%%%%%%%%%%%%%%%%%%%%%%%%%%%%
% This is FULLPAGE.STY by H.Partl, Version 2 as of 15 Dec 1988.
% Document Style Option to fill the paper just like Plain TeX.

\typeout{Style Option FULLPAGE Version 2 as of 15 Dec 1988}

\topmargin 0pt
\advance \topmargin by -\headheight
\advance \topmargin by -\headsep

\textheight 8.9in

\oddsidemargin 0pt
\evensidemargin \oddsidemargin
\marginparwidth 0.5in

\textwidth 6.5in
%%%%%%%%%%%%%%%%%%%%%%%%%%%%%%%%%%%%%%%%%%%%%%%%%%%%%%%%%%%%%%%%

\pagestyle{empty}
\setlength{\oddsidemargin}{0in}
\setlength{\topmargin}{-0.8in}
\setlength{\textwidth}{6.8in}
\setlength{\textheight}{9.5in}


\def\ind{\hspace*{0.3in}}
\def\gap{0.1in}
\def\bigap{0.25in}
\newcommand{\Xomit}[1]{}
\newcommand{\Mod}[1]{\ (\mathrm{mod}\ #1)}


\begin{document}

\setlength{\parindent}{0in}
\addtolength{\parskip}{0.1cm}
\setlength{\fboxrule}{.5mm}\setlength{\fboxsep}{1.2mm}
\newlength{\boxlength}\setlength{\boxlength}{\textwidth}
\addtolength{\boxlength}{-4mm}
\begin{center}\framebox{\parbox{\boxlength}{{\bf
CS 4820, Spring 2019 \hfill Homework 5, Problem 1}\\
% TODO: fill in your own name, netID, and collaborators
Name: \\
NetID: \\
}}
\end{center}
\vspace{5mm}
\vskip \bigap
{\bf (1a)} {\em (5 points)}
Design a dynamic programming algorithm that computes,
for every pair $i,j$, the quantity $q_{ij} =
\Pr(X_1+\cdots+X_i = j)$, and then outputs
the sum $\sum_{j=a}^b q_{kj}$. \\[6pt] {\bf You may omit
the proof of correctness, but you should analyze the
running time of this dynamic programming algorithm.}

\vskip \gap
{\bf (1b)} {\em (10 points)}
Design an algorithm
that computes $\Pr(a \le X \le b)$
in time $O(k \, n \log (k) \log (kn))$. \\[6pt]
{\bf For this part of the problem, include both the
running time analysis and the proof of correctness.}

{\em HINT: If $Y$ and $Z$ are independent random
variables taking values in $\{0,1,\ldots,n-1\}$,
show that the probability distribution of their
sum $Y+Z$ can be computed as the convolution
of two vectors representing the probability
distributions of $Y$ and of $Z$.}

\vskip \bigap

%% Your solution goes here.


\end{document}
