\documentclass[]{article}
\usepackage{lmodern}
\usepackage{amssymb,amsmath}
\usepackage{ifxetex,ifluatex}
\usepackage{fixltx2e} % provides \textsubscript
\ifnum 0\ifxetex 1\fi\ifluatex 1\fi=0 % if pdftex
  \usepackage[T1]{fontenc}
  \usepackage[utf8]{inputenc}
\else % if luatex or xelatex
  \ifxetex
    \usepackage{mathspec}
  \else
    \usepackage{fontspec}
  \fi
  \defaultfontfeatures{Ligatures=TeX,Scale=MatchLowercase}
\fi
% use upquote if available, for straight quotes in verbatim environments
\IfFileExists{upquote.sty}{\usepackage{upquote}}{}
% use microtype if available
\IfFileExists{microtype.sty}{%
\usepackage{microtype}
\UseMicrotypeSet[protrusion]{basicmath} % disable protrusion for tt fonts
}{}
\usepackage[margin=1in]{geometry}
\usepackage{hyperref}
\hypersetup{unicode=true,
            pdftitle={BTRY 3020/StSCI 3200 Homework VI},
            pdfborder={0 0 0},
            breaklinks=true}
\urlstyle{same}  % don't use monospace font for urls
\usepackage{color}
\usepackage{fancyvrb}
\newcommand{\VerbBar}{|}
\newcommand{\VERB}{\Verb[commandchars=\\\{\}]}
\DefineVerbatimEnvironment{Highlighting}{Verbatim}{commandchars=\\\{\}}
% Add ',fontsize=\small' for more characters per line
\usepackage{framed}
\definecolor{shadecolor}{RGB}{248,248,248}
\newenvironment{Shaded}{\begin{snugshade}}{\end{snugshade}}
\newcommand{\KeywordTok}[1]{\textcolor[rgb]{0.13,0.29,0.53}{\textbf{#1}}}
\newcommand{\DataTypeTok}[1]{\textcolor[rgb]{0.13,0.29,0.53}{#1}}
\newcommand{\DecValTok}[1]{\textcolor[rgb]{0.00,0.00,0.81}{#1}}
\newcommand{\BaseNTok}[1]{\textcolor[rgb]{0.00,0.00,0.81}{#1}}
\newcommand{\FloatTok}[1]{\textcolor[rgb]{0.00,0.00,0.81}{#1}}
\newcommand{\ConstantTok}[1]{\textcolor[rgb]{0.00,0.00,0.00}{#1}}
\newcommand{\CharTok}[1]{\textcolor[rgb]{0.31,0.60,0.02}{#1}}
\newcommand{\SpecialCharTok}[1]{\textcolor[rgb]{0.00,0.00,0.00}{#1}}
\newcommand{\StringTok}[1]{\textcolor[rgb]{0.31,0.60,0.02}{#1}}
\newcommand{\VerbatimStringTok}[1]{\textcolor[rgb]{0.31,0.60,0.02}{#1}}
\newcommand{\SpecialStringTok}[1]{\textcolor[rgb]{0.31,0.60,0.02}{#1}}
\newcommand{\ImportTok}[1]{#1}
\newcommand{\CommentTok}[1]{\textcolor[rgb]{0.56,0.35,0.01}{\textit{#1}}}
\newcommand{\DocumentationTok}[1]{\textcolor[rgb]{0.56,0.35,0.01}{\textbf{\textit{#1}}}}
\newcommand{\AnnotationTok}[1]{\textcolor[rgb]{0.56,0.35,0.01}{\textbf{\textit{#1}}}}
\newcommand{\CommentVarTok}[1]{\textcolor[rgb]{0.56,0.35,0.01}{\textbf{\textit{#1}}}}
\newcommand{\OtherTok}[1]{\textcolor[rgb]{0.56,0.35,0.01}{#1}}
\newcommand{\FunctionTok}[1]{\textcolor[rgb]{0.00,0.00,0.00}{#1}}
\newcommand{\VariableTok}[1]{\textcolor[rgb]{0.00,0.00,0.00}{#1}}
\newcommand{\ControlFlowTok}[1]{\textcolor[rgb]{0.13,0.29,0.53}{\textbf{#1}}}
\newcommand{\OperatorTok}[1]{\textcolor[rgb]{0.81,0.36,0.00}{\textbf{#1}}}
\newcommand{\BuiltInTok}[1]{#1}
\newcommand{\ExtensionTok}[1]{#1}
\newcommand{\PreprocessorTok}[1]{\textcolor[rgb]{0.56,0.35,0.01}{\textit{#1}}}
\newcommand{\AttributeTok}[1]{\textcolor[rgb]{0.77,0.63,0.00}{#1}}
\newcommand{\RegionMarkerTok}[1]{#1}
\newcommand{\InformationTok}[1]{\textcolor[rgb]{0.56,0.35,0.01}{\textbf{\textit{#1}}}}
\newcommand{\WarningTok}[1]{\textcolor[rgb]{0.56,0.35,0.01}{\textbf{\textit{#1}}}}
\newcommand{\AlertTok}[1]{\textcolor[rgb]{0.94,0.16,0.16}{#1}}
\newcommand{\ErrorTok}[1]{\textcolor[rgb]{0.64,0.00,0.00}{\textbf{#1}}}
\newcommand{\NormalTok}[1]{#1}
\usepackage{graphicx,grffile}
\makeatletter
\def\maxwidth{\ifdim\Gin@nat@width>\linewidth\linewidth\else\Gin@nat@width\fi}
\def\maxheight{\ifdim\Gin@nat@height>\textheight\textheight\else\Gin@nat@height\fi}
\makeatother
% Scale images if necessary, so that they will not overflow the page
% margins by default, and it is still possible to overwrite the defaults
% using explicit options in \includegraphics[width, height, ...]{}
\setkeys{Gin}{width=\maxwidth,height=\maxheight,keepaspectratio}
\IfFileExists{parskip.sty}{%
\usepackage{parskip}
}{% else
\setlength{\parindent}{0pt}
\setlength{\parskip}{6pt plus 2pt minus 1pt}
}
\setlength{\emergencystretch}{3em}  % prevent overfull lines
\providecommand{\tightlist}{%
  \setlength{\itemsep}{0pt}\setlength{\parskip}{0pt}}
\setcounter{secnumdepth}{0}
% Redefines (sub)paragraphs to behave more like sections
\ifx\paragraph\undefined\else
\let\oldparagraph\paragraph
\renewcommand{\paragraph}[1]{\oldparagraph{#1}\mbox{}}
\fi
\ifx\subparagraph\undefined\else
\let\oldsubparagraph\subparagraph
\renewcommand{\subparagraph}[1]{\oldsubparagraph{#1}\mbox{}}
\fi

%%% Use protect on footnotes to avoid problems with footnotes in titles
\let\rmarkdownfootnote\footnote%
\def\footnote{\protect\rmarkdownfootnote}

%%% Change title format to be more compact
\usepackage{titling}

% Create subtitle command for use in maketitle
\newcommand{\subtitle}[1]{
  \posttitle{
    \begin{center}\large#1\end{center}
    }
}

\setlength{\droptitle}{-2em}

  \title{BTRY 3020/StSCI 3200 Homework VI}
    \pretitle{\vspace{\droptitle}\centering\huge}
  \posttitle{\par}
    \author{}
    \preauthor{}\postauthor{}
    \date{}
    \predate{}\postdate{}
  

\begin{document}
\maketitle

\section{NAME: Kevin Klaben}\label{name-kevin-klaben}

\section{NETID: kek228}\label{netid-kek228}

\subsection{\texorpdfstring{\#\textbf{DUE DATE: 8:40 am Tuesday March 26
2019}}{\#DUE DATE: 8:40 am Tuesday March 26 2019}}\label{due-date-840-am-tuesday-march-26-2019}

\section{Question 1.}\label{question-1.}

Health officials wonder why some people get the flu shot while others
don't. In a study designed to shed some light on this, researchers asked
a random sample of patients if they had gotten aflu shot, recorded their
age and gender, and also gave each a written questionnaire designed to
evaluate their health awareness index (Ind). Data appear in
Hwk6Q1DatSp19. Note here that Y = 1 means they received the flu shot and
that males were coded as \(X_3\) = 1, females coded as \(X_3\) = 0.

\begin{enumerate}
\def\labelenumi{\Alph{enumi})}
\tightlist
\item
  Obtain the maximum likelihood estimators of \(\beta_o\), \(\beta_1\),
  \(\beta_2\), and \(\beta_3\). State the fitted regression function.
\end{enumerate}

\begin{Shaded}
\begin{Highlighting}[]
\KeywordTok{library}\NormalTok{(readxl)}
\NormalTok{HealthDat <-}\StringTok{ }\KeywordTok{read_excel}\NormalTok{(}\StringTok{"Hwk6Q1DatSp19.xlsx"}\NormalTok{)}
\NormalTok{HealthDat}\OperatorTok{$}\NormalTok{Gen=}\KeywordTok{factor}\NormalTok{(HealthDat}\OperatorTok{$}\NormalTok{Gen)}
\NormalTok{HealthDat}\OperatorTok{$}\NormalTok{FShot=}\KeywordTok{factor}\NormalTok{(HealthDat}\OperatorTok{$}\NormalTok{FShot)}

\NormalTok{flu.glm=}\KeywordTok{glm}\NormalTok{(FShot}\OperatorTok{~}\NormalTok{Age}\OperatorTok{+}\NormalTok{Ind}\OperatorTok{+}\NormalTok{Gen,}\DataTypeTok{family=}\NormalTok{binomial,}\DataTypeTok{data=}\NormalTok{HealthDat)}

\KeywordTok{summary}\NormalTok{(flu.glm)}
\end{Highlighting}
\end{Shaded}

\begin{verbatim}
## 
## Call:
## glm(formula = FShot ~ Age + Ind + Gen, family = binomial, data = HealthDat)
## 
## Deviance Residuals: 
##     Min       1Q   Median       3Q      Max  
## -1.4037  -0.5637  -0.3352  -0.1542   2.9394  
## 
## Coefficients:
##             Estimate Std. Error z value Pr(>|z|)   
## (Intercept) -1.17716    2.98242  -0.395  0.69307   
## Age          0.07279    0.03038   2.396  0.01658 * 
## Ind         -0.09899    0.03348  -2.957  0.00311 **
## Gen1         0.43397    0.52179   0.832  0.40558   
## ---
## Signif. codes:  0 '***' 0.001 '**' 0.01 '*' 0.05 '.' 0.1 ' ' 1
## 
## (Dispersion parameter for binomial family taken to be 1)
## 
##     Null deviance: 134.94  on 158  degrees of freedom
## Residual deviance: 105.09  on 155  degrees of freedom
## AIC: 113.09
## 
## Number of Fisher Scoring iterations: 6
\end{verbatim}

\begin{Shaded}
\begin{Highlighting}[]
\KeywordTok{plot}\NormalTok{(flu.glm, }\DataTypeTok{which=}\DecValTok{4}\NormalTok{)}
\end{Highlighting}
\end{Shaded}

\includegraphics{Hwk6FinDrtSp19_files/figure-latex/unnamed-chunk-1-1.pdf}

\(\beta_o\) = -1.77 \(\beta_1\) = 0.0728 \(\beta_2\) = -0.099
\(\beta_3\) = 0.434 The fitted regression is: log(odds of having gotten
flu shot)= -1.77+0.0728(Age)-0.099(Ind)+0.434(Gen)

\begin{enumerate}
\def\labelenumi{\Alph{enumi})}
\setcounter{enumi}{1}
\tightlist
\item
  What is the estimated probability (point estimate) of getting the flu
  shot that a male clients aged 55 years with a health awareness score
  of 60?
\end{enumerate}

Age=55 Ind=60 Gen=1 pt-est of log(odds of gotten flu
shot)=-1.77+0.0728(55)-0.099(60)+0.434(1)

\begin{Shaded}
\begin{Highlighting}[]
\OperatorTok{-}\FloatTok{1.77}\OperatorTok{+}\FloatTok{0.0728}\OperatorTok{*}\NormalTok{(}\DecValTok{55}\NormalTok{)}\OperatorTok{-}\FloatTok{0.099}\OperatorTok{*}\NormalTok{(}\DecValTok{60}\NormalTok{)}\OperatorTok{+}\FloatTok{0.434}\OperatorTok{*}\NormalTok{(}\DecValTok{1}\NormalTok{)}
\end{Highlighting}
\end{Shaded}

\begin{verbatim}
## [1] -3.272
\end{verbatim}

\begin{Shaded}
\begin{Highlighting}[]
\KeywordTok{exp}\NormalTok{(}\OperatorTok{-}\FloatTok{3.272}\NormalTok{)}
\end{Highlighting}
\end{Shaded}

\begin{verbatim}
## [1] 0.03793049
\end{verbatim}

Thus, the pt. estimate for the odds of a 55 year old male with a health
awareness score is predicted to be on average 0.0379.

\begin{enumerate}
\def\labelenumi{\Alph{enumi})}
\setcounter{enumi}{2}
\tightlist
\item
  Obtain the VIFs for the regression predictors. What conclusions can
  you reach from these statistics?
\end{enumerate}

\begin{Shaded}
\begin{Highlighting}[]
\KeywordTok{library}\NormalTok{(car)}
\end{Highlighting}
\end{Shaded}

\begin{verbatim}
## Loading required package: carData
\end{verbatim}

\begin{Shaded}
\begin{Highlighting}[]
\KeywordTok{vif}\NormalTok{(flu.glm)}
\end{Highlighting}
\end{Shaded}

\begin{verbatim}
##      Age      Ind      Gen 
## 1.091111 1.081049 1.048432
\end{verbatim}

The VIFs are all less than 4 and thus, we can conclude that
interpretting the coefficients of the parameters is valid and out
estimates will be valid.

\begin{enumerate}
\def\labelenumi{\Alph{enumi})}
\setcounter{enumi}{3}
\tightlist
\item
  Get the standardized deviance residuals and plot against observation
  number. Are there any outliers?
\end{enumerate}

\begin{Shaded}
\begin{Highlighting}[]
\NormalTok{res <-}\StringTok{ }\KeywordTok{residuals}\NormalTok{(flu.glm, }\DataTypeTok{type =} \StringTok{"deviance"}\NormalTok{)}
\KeywordTok{plot}\NormalTok{(HealthDat}\OperatorTok{$}\NormalTok{ObsNum, res,}\DataTypeTok{xlab=}\StringTok{"Observation Number"}\NormalTok{, }\DataTypeTok{ylab =} \StringTok{"Deviance Residuals"}\NormalTok{)}
\KeywordTok{abline}\NormalTok{(}\DataTypeTok{h =} \DecValTok{0}\NormalTok{)}
\end{Highlighting}
\end{Shaded}

\includegraphics{Hwk6FinDrtSp19_files/figure-latex/unnamed-chunk-4-1.pdf}
Yes, there is one point which sticks out and have large deviance
residual, particularly the one observation that appears to have
observation number just under 50 has a deviance residual much higher
than the rest.

\begin{enumerate}
\def\labelenumi{\Alph{enumi})}
\setcounter{enumi}{4}
\tightlist
\item
  Get the Cook's distance numbers and plot against observation number.
  Do there appear to be any influential outliers? If so, check their
  effects.
\end{enumerate}

\begin{Shaded}
\begin{Highlighting}[]
\KeywordTok{plot}\NormalTok{(flu.glm, }\DataTypeTok{which=}\DecValTok{4}\NormalTok{)}
\end{Highlighting}
\end{Shaded}

\includegraphics{Hwk6FinDrtSp19_files/figure-latex/unnamed-chunk-5-1.pdf}
It appears that observation 47 is more influenctial than most other
points.

\begin{Shaded}
\begin{Highlighting}[]
\KeywordTok{summary}\NormalTok{(flu.glm)}
\end{Highlighting}
\end{Shaded}

\begin{verbatim}
## 
## Call:
## glm(formula = FShot ~ Age + Ind + Gen, family = binomial, data = HealthDat)
## 
## Deviance Residuals: 
##     Min       1Q   Median       3Q      Max  
## -1.4037  -0.5637  -0.3352  -0.1542   2.9394  
## 
## Coefficients:
##             Estimate Std. Error z value Pr(>|z|)   
## (Intercept) -1.17716    2.98242  -0.395  0.69307   
## Age          0.07279    0.03038   2.396  0.01658 * 
## Ind         -0.09899    0.03348  -2.957  0.00311 **
## Gen1         0.43397    0.52179   0.832  0.40558   
## ---
## Signif. codes:  0 '***' 0.001 '**' 0.01 '*' 0.05 '.' 0.1 ' ' 1
## 
## (Dispersion parameter for binomial family taken to be 1)
## 
##     Null deviance: 134.94  on 158  degrees of freedom
## Residual deviance: 105.09  on 155  degrees of freedom
## AIC: 113.09
## 
## Number of Fisher Scoring iterations: 6
\end{verbatim}

\begin{Shaded}
\begin{Highlighting}[]
\NormalTok{HealthDat2=HealthDat[}\OperatorTok{-}\DecValTok{31}\NormalTok{,]}
\NormalTok{flu2.glm=}\KeywordTok{glm}\NormalTok{(FShot}\OperatorTok{~}\NormalTok{Age}\OperatorTok{+}\NormalTok{Ind}\OperatorTok{+}\NormalTok{Gen,}\DataTypeTok{family=}\NormalTok{binomial,}\DataTypeTok{data=}\NormalTok{HealthDat2)}
\KeywordTok{summary}\NormalTok{(flu2.glm)}
\end{Highlighting}
\end{Shaded}

\begin{verbatim}
## 
## Call:
## glm(formula = FShot ~ Age + Ind + Gen, family = binomial, data = HealthDat2)
## 
## Deviance Residuals: 
##     Min       1Q   Median       3Q      Max  
## -1.4018  -0.5651  -0.3358  -0.1543   2.9341  
## 
## Coefficients:
##             Estimate Std. Error z value Pr(>|z|)   
## (Intercept) -1.19603    2.97836  -0.402   0.6880   
## Age          0.07257    0.03034   2.392   0.0168 * 
## Ind         -0.09835    0.03347  -2.938   0.0033 **
## Gen1         0.43984    0.52140   0.844   0.3989   
## ---
## Signif. codes:  0 '***' 0.001 '**' 0.01 '*' 0.05 '.' 0.1 ' ' 1
## 
## (Dispersion parameter for binomial family taken to be 1)
## 
##     Null deviance: 134.61  on 157  degrees of freedom
## Residual deviance: 104.97  on 154  degrees of freedom
## AIC: 112.97
## 
## Number of Fisher Scoring iterations: 6
\end{verbatim}

The dropping of this outlier has little affect on the significance of
the predictors and little affect on the values of the coefficients.
Thus, we will continue with it in our model.

\begin{enumerate}
\def\labelenumi{\Alph{enumi})}
\setcounter{enumi}{5}
\tightlist
\item
  Can we drop Age and Gender if we keep the health awareness index in
  the model? State hypotheses, test statistic, p-value, and conclusions.
\end{enumerate}

Ho: \(\beta_1\)=\(\beta_3\)=0 Ha: Not Ho

\begin{Shaded}
\begin{Highlighting}[]
\KeywordTok{summary}\NormalTok{(flu.glm)}
\end{Highlighting}
\end{Shaded}

\begin{verbatim}
## 
## Call:
## glm(formula = FShot ~ Age + Ind + Gen, family = binomial, data = HealthDat)
## 
## Deviance Residuals: 
##     Min       1Q   Median       3Q      Max  
## -1.4037  -0.5637  -0.3352  -0.1542   2.9394  
## 
## Coefficients:
##             Estimate Std. Error z value Pr(>|z|)   
## (Intercept) -1.17716    2.98242  -0.395  0.69307   
## Age          0.07279    0.03038   2.396  0.01658 * 
## Ind         -0.09899    0.03348  -2.957  0.00311 **
## Gen1         0.43397    0.52179   0.832  0.40558   
## ---
## Signif. codes:  0 '***' 0.001 '**' 0.01 '*' 0.05 '.' 0.1 ' ' 1
## 
## (Dispersion parameter for binomial family taken to be 1)
## 
##     Null deviance: 134.94  on 158  degrees of freedom
## Residual deviance: 105.09  on 155  degrees of freedom
## AIC: 113.09
## 
## Number of Fisher Scoring iterations: 6
\end{verbatim}

\begin{Shaded}
\begin{Highlighting}[]
\NormalTok{flured.glm=}\KeywordTok{glm}\NormalTok{(FShot}\OperatorTok{~}\NormalTok{Ind,}\DataTypeTok{family=}\NormalTok{binomial,}\DataTypeTok{data=}\NormalTok{HealthDat)}
\KeywordTok{summary}\NormalTok{(flured.glm)}
\end{Highlighting}
\end{Shaded}

\begin{verbatim}
## 
## Call:
## glm(formula = FShot ~ Ind, family = binomial, data = HealthDat)
## 
## Deviance Residuals: 
##     Min       1Q   Median       3Q      Max  
## -1.3944  -0.5926  -0.3999  -0.2369   2.8476  
## 
## Coefficients:
##             Estimate Std. Error z value Pr(>|z|)    
## (Intercept)  4.91133    1.62651    3.02  0.00253 ** 
## Ind         -0.11931    0.03013   -3.96  7.5e-05 ***
## ---
## Signif. codes:  0 '***' 0.001 '**' 0.01 '*' 0.05 '.' 0.1 ' ' 1
## 
## (Dispersion parameter for binomial family taken to be 1)
## 
##     Null deviance: 134.94  on 158  degrees of freedom
## Residual deviance: 113.20  on 157  degrees of freedom
## AIC: 117.2
## 
## Number of Fisher Scoring iterations: 5
\end{verbatim}

Ho:β1=β3=0 Ha:Not Ho Dev(flured.glm)-Dev(flu.glm)=113.20-105.09=8.11
p=P(\(X^2_2\)\textgreater{}8.11)

\begin{Shaded}
\begin{Highlighting}[]
\DecValTok{1}\OperatorTok{-}\KeywordTok{pchisq}\NormalTok{(}\FloatTok{8.11}\NormalTok{,}\DecValTok{2}\NormalTok{)}
\end{Highlighting}
\end{Shaded}

\begin{verbatim}
## [1] 0.01733548
\end{verbatim}

As p=0.017335\textless{}Alpha=0.05, we reject the null hypothesis and
conclude that at least one of the Gen or Age terms is signifcant and
thus they cannot both be dropped.

\begin{enumerate}
\def\labelenumi{\Alph{enumi})}
\setcounter{enumi}{6}
\tightlist
\item
  Install the package ``bestglm''. Visit the following website:
\end{enumerate}

\url{https://cran.r-project.org/web/packages/bestglm/vignettes/bestglm.pdf}

to learn how to use this package. Don't forget the ``library(bestglm)''
command before you use it. (The unusual thing about bestglm is the
creation of an ``XY'' matrix before you use it. This is a matrix where
the first columns are your candidate predictors (X's) and the last is
your response (Y). This amounts to a reordering of the columns in your
dataset, with the Y last, and omitting any columns you don't want to use
as predictors (like ``Observation Number''.) Then give the command ``XY
= data.frame(XY)'' before running bestglm).

\begin{enumerate}
\def\labelenumi{\roman{enumi})}
\tightlist
\item
  Find the best model for getting a flu shot according to the BIC
  criteria
\end{enumerate}

\begin{Shaded}
\begin{Highlighting}[]
\KeywordTok{library}\NormalTok{(bestglm)}
\end{Highlighting}
\end{Shaded}

\begin{verbatim}
## Loading required package: leaps
\end{verbatim}

\begin{Shaded}
\begin{Highlighting}[]
\NormalTok{HealthDatXY <-}\StringTok{ }\NormalTok{HealthDat[}\KeywordTok{c}\NormalTok{(}\StringTok{"Age"}\NormalTok{, }\StringTok{"Ind"}\NormalTok{, }\StringTok{"Gen"}\NormalTok{, }\StringTok{"FShot"}\NormalTok{)]}
\NormalTok{XY=}\KeywordTok{data.frame}\NormalTok{(HealthDatXY)}
\NormalTok{XY}\OperatorTok{$}\NormalTok{Gen=}\KeywordTok{factor}\NormalTok{(XY}\OperatorTok{$}\NormalTok{Gen)}
\NormalTok{XY}\OperatorTok{$}\NormalTok{FShot=}\KeywordTok{factor}\NormalTok{(XY}\OperatorTok{$}\NormalTok{FShot)}
\NormalTok{best.glm=}\KeywordTok{bestglm}\NormalTok{(XY, }\DataTypeTok{IC=}\StringTok{"BIC"}\NormalTok{, }\DataTypeTok{family=}\NormalTok{binomial)}
\end{Highlighting}
\end{Shaded}

\begin{verbatim}
## Morgan-Tatar search since family is non-gaussian.
\end{verbatim}

The best model according to BIC criteria uses Age and Ind with the
coefficients as in the output. log(odds of
disease)=-1.458+0.079(Age)-0.095(Ind) ii) Find the best models for a 0,
1, 2, and 3 predictors using the Subsets command

\begin{Shaded}
\begin{Highlighting}[]
\KeywordTok{library}\NormalTok{(leaps)}

\NormalTok{Model <-}\StringTok{ }\KeywordTok{regsubsets}\NormalTok{(FShot}\OperatorTok{~}\NormalTok{Age}\OperatorTok{+}\NormalTok{Ind}\OperatorTok{+}\NormalTok{Gen, }\DataTypeTok{data=}\NormalTok{HealthDat, }\DataTypeTok{nbest=}\DecValTok{1}\NormalTok{)}
\KeywordTok{summary}\NormalTok{(Model)}
\end{Highlighting}
\end{Shaded}

\begin{verbatim}
## Subset selection object
## Call: regsubsets.formula(FShot ~ Age + Ind + Gen, data = HealthDat, 
##     nbest = 1)
## 3 Variables  (and intercept)
##      Forced in Forced out
## Age      FALSE      FALSE
## Ind      FALSE      FALSE
## Gen1     FALSE      FALSE
## 1 subsets of each size up to 3
## Selection Algorithm: exhaustive
##          Age Ind Gen1
## 1  ( 1 ) " " "*" " " 
## 2  ( 1 ) "*" "*" " " 
## 3  ( 1 ) "*" "*" "*"
\end{verbatim}

\begin{Shaded}
\begin{Highlighting}[]
\NormalTok{fluzero.glm=}\KeywordTok{glm}\NormalTok{(FShot}\OperatorTok{~}\DecValTok{1}\NormalTok{,}\DataTypeTok{family=}\NormalTok{binomial,}\DataTypeTok{data=}\NormalTok{HealthDat)}
\KeywordTok{summary}\NormalTok{(fluzero.glm)}
\end{Highlighting}
\end{Shaded}

\begin{verbatim}
## 
## Call:
## glm(formula = FShot ~ 1, family = binomial, data = HealthDat)
## 
## Deviance Residuals: 
##     Min       1Q   Median       3Q      Max  
## -0.5721  -0.5721  -0.5721  -0.5721   1.9447  
## 
## Coefficients:
##             Estimate Std. Error z value Pr(>|z|)    
## (Intercept)  -1.7272     0.2215  -7.797 6.34e-15 ***
## ---
## Signif. codes:  0 '***' 0.001 '**' 0.01 '*' 0.05 '.' 0.1 ' ' 1
## 
## (Dispersion parameter for binomial family taken to be 1)
## 
##     Null deviance: 134.94  on 158  degrees of freedom
## Residual deviance: 134.94  on 158  degrees of freedom
## AIC: 136.94
## 
## Number of Fisher Scoring iterations: 4
\end{verbatim}

\begin{Shaded}
\begin{Highlighting}[]
\NormalTok{fluone.glm=}\KeywordTok{glm}\NormalTok{(FShot}\OperatorTok{~}\NormalTok{Ind,}\DataTypeTok{family=}\NormalTok{binomial,}\DataTypeTok{data=}\NormalTok{HealthDat)}
\KeywordTok{summary}\NormalTok{(fluone.glm)}
\end{Highlighting}
\end{Shaded}

\begin{verbatim}
## 
## Call:
## glm(formula = FShot ~ Ind, family = binomial, data = HealthDat)
## 
## Deviance Residuals: 
##     Min       1Q   Median       3Q      Max  
## -1.3944  -0.5926  -0.3999  -0.2369   2.8476  
## 
## Coefficients:
##             Estimate Std. Error z value Pr(>|z|)    
## (Intercept)  4.91133    1.62651    3.02  0.00253 ** 
## Ind         -0.11931    0.03013   -3.96  7.5e-05 ***
## ---
## Signif. codes:  0 '***' 0.001 '**' 0.01 '*' 0.05 '.' 0.1 ' ' 1
## 
## (Dispersion parameter for binomial family taken to be 1)
## 
##     Null deviance: 134.94  on 158  degrees of freedom
## Residual deviance: 113.20  on 157  degrees of freedom
## AIC: 117.2
## 
## Number of Fisher Scoring iterations: 5
\end{verbatim}

\begin{Shaded}
\begin{Highlighting}[]
\NormalTok{flutwo.glm=}\KeywordTok{glm}\NormalTok{(FShot}\OperatorTok{~}\NormalTok{Age}\OperatorTok{+}\NormalTok{Ind,}\DataTypeTok{family=}\NormalTok{binomial,}\DataTypeTok{data=}\NormalTok{HealthDat)}
\KeywordTok{summary}\NormalTok{(flutwo.glm)}
\end{Highlighting}
\end{Shaded}

\begin{verbatim}
## 
## Call:
## glm(formula = FShot ~ Age + Ind, family = binomial, data = HealthDat)
## 
## Deviance Residuals: 
##     Min       1Q   Median       3Q      Max  
## -1.4479  -0.5708  -0.3390  -0.1629   2.8430  
## 
## Coefficients:
##             Estimate Std. Error z value Pr(>|z|)   
## (Intercept) -1.45778    2.91534  -0.500  0.61705   
## Age          0.07787    0.02970   2.622  0.00873 **
## Ind         -0.09547    0.03241  -2.946  0.00322 **
## ---
## Signif. codes:  0 '***' 0.001 '**' 0.01 '*' 0.05 '.' 0.1 ' ' 1
## 
## (Dispersion parameter for binomial family taken to be 1)
## 
##     Null deviance: 134.94  on 158  degrees of freedom
## Residual deviance: 105.80  on 156  degrees of freedom
## AIC: 111.8
## 
## Number of Fisher Scoring iterations: 6
\end{verbatim}

\begin{Shaded}
\begin{Highlighting}[]
\NormalTok{fluthree.glm=}\KeywordTok{glm}\NormalTok{(FShot}\OperatorTok{~}\NormalTok{Age}\OperatorTok{+}\NormalTok{Ind}\OperatorTok{+}\NormalTok{Gen,}\DataTypeTok{family=}\NormalTok{binomial,}\DataTypeTok{data=}\NormalTok{HealthDat)}
\KeywordTok{summary}\NormalTok{(fluthree.glm)}
\end{Highlighting}
\end{Shaded}

\begin{verbatim}
## 
## Call:
## glm(formula = FShot ~ Age + Ind + Gen, family = binomial, data = HealthDat)
## 
## Deviance Residuals: 
##     Min       1Q   Median       3Q      Max  
## -1.4037  -0.5637  -0.3352  -0.1542   2.9394  
## 
## Coefficients:
##             Estimate Std. Error z value Pr(>|z|)   
## (Intercept) -1.17716    2.98242  -0.395  0.69307   
## Age          0.07279    0.03038   2.396  0.01658 * 
## Ind         -0.09899    0.03348  -2.957  0.00311 **
## Gen1         0.43397    0.52179   0.832  0.40558   
## ---
## Signif. codes:  0 '***' 0.001 '**' 0.01 '*' 0.05 '.' 0.1 ' ' 1
## 
## (Dispersion parameter for binomial family taken to be 1)
## 
##     Null deviance: 134.94  on 158  degrees of freedom
## Residual deviance: 105.09  on 155  degrees of freedom
## AIC: 113.09
## 
## Number of Fisher Scoring iterations: 6
\end{verbatim}

0 Predictor model= log(odds of disease)=-1.7272 1 predictor model uses
Ind: log(odds of disease)=4.911-0.119(Ind) 2 predictor model uses Age
and Ind: log(odds of disease)=-1.458+0.079(Age)-0.095(Ind) 3 predictor
model uses Age, Ind, and Gen: log(odds of
disease)=-1.177+0.0729(Age)-0.099(Ind)+0.434(Gen1) iii) Find the best
model for getting a flu shot according to the AIC criteria

\begin{Shaded}
\begin{Highlighting}[]
\KeywordTok{library}\NormalTok{(MASS)}
\NormalTok{null=}\KeywordTok{glm}\NormalTok{(FShot}\OperatorTok{~}\DecValTok{1}\NormalTok{, }\DataTypeTok{family=}\NormalTok{binomial, }\DataTypeTok{data=}\NormalTok{HealthDat)}
\NormalTok{full=}\KeywordTok{glm}\NormalTok{(FShot}\OperatorTok{~}\NormalTok{Age}\OperatorTok{+}\NormalTok{Ind}\OperatorTok{+}\NormalTok{Gen, }\DataTypeTok{family=}\NormalTok{binomial, }\DataTypeTok{data=}\NormalTok{HealthDat)}
\KeywordTok{step}\NormalTok{(full, }\DataTypeTok{data=}\NormalTok{HealthDat, }\DataTypeTok{direction=}\StringTok{"backward"}\NormalTok{)}
\end{Highlighting}
\end{Shaded}

\begin{verbatim}
## Start:  AIC=113.09
## FShot ~ Age + Ind + Gen
## 
##        Df Deviance    AIC
## - Gen   1   105.80 111.80
## <none>      105.09 113.09
## - Age   1   111.19 117.19
## - Ind   1   115.80 121.80
## 
## Step:  AIC=111.8
## FShot ~ Age + Ind
## 
##        Df Deviance    AIC
## <none>      105.80 111.80
## - Age   1   113.20 117.20
## - Ind   1   116.27 120.27
\end{verbatim}

\begin{verbatim}
## 
## Call:  glm(formula = FShot ~ Age + Ind, family = binomial, data = HealthDat)
## 
## Coefficients:
## (Intercept)          Age          Ind  
##    -1.45778      0.07787     -0.09547  
## 
## Degrees of Freedom: 158 Total (i.e. Null);  156 Residual
## Null Deviance:       134.9 
## Residual Deviance: 105.8     AIC: 111.8
\end{verbatim}

According to AIC the best model uses predictors Age and Ind as follows:
log(odds of disease)=-1.458+0.079(Age)-0.095(Ind)

\begin{enumerate}
\def\labelenumi{\roman{enumi})}
\setcounter{enumi}{3}
\tightlist
\item
  Find the best models for a 0, 1, 2, and 3 predictors using the Subsets
  command
\end{enumerate}

\begin{Shaded}
\begin{Highlighting}[]
\KeywordTok{library}\NormalTok{(bestglm)}
\NormalTok{best2.glm=}\KeywordTok{bestglm}\NormalTok{(XY, }\DataTypeTok{IC=}\StringTok{"AIC"}\NormalTok{, }\DataTypeTok{family=}\NormalTok{binomial)}
\end{Highlighting}
\end{Shaded}

\begin{verbatim}
## Morgan-Tatar search since family is non-gaussian.
\end{verbatim}

\begin{Shaded}
\begin{Highlighting}[]
\NormalTok{best2.glm}\OperatorTok{$}\NormalTok{Subsets}
\end{Highlighting}
\end{Shaded}

\begin{verbatim}
##    Intercept   Age   Ind   Gen logLikelihood      AIC
## 0       TRUE FALSE FALSE FALSE     -67.47038 134.9408
## 1       TRUE FALSE  TRUE FALSE     -56.59790 115.1958
## 2*      TRUE  TRUE  TRUE FALSE     -52.89769 109.7954
## 3       TRUE  TRUE  TRUE  TRUE     -52.54659 111.0932
\end{verbatim}

The best models using AIC for each number of predictors 0-3 produces the
same best models as were listed above in part (ii).

\begin{enumerate}
\def\labelenumi{\alph{enumi})}
\setcounter{enumi}{21}
\tightlist
\item
  What model from the above models evaluated would you choose for this
  situation? Explain BRIEFLY; you may include data from all parts of
  Question 1.
\end{enumerate}

FShot \textasciitilde{} Age + Ind

As both AIC which tends to overfit and BIC which tends to underfit
selected these predictors as the best overall model, we will choose this
as our best possible model to use as it is agreed upon by both criteria.

The final model is log(odds of disease)=-1.458+0.079(Age)-0.095(Ind)

\begin{enumerate}
\def\labelenumi{\roman{enumi})}
\setcounter{enumi}{5}
\tightlist
\item
  Before implementing this model, what should you try to do to enhance
  it?
\end{enumerate}

Add polynomial terms and test their significance

\begin{Shaded}
\begin{Highlighting}[]
\NormalTok{HealthDat}\OperatorTok{$}\NormalTok{AgeSq=HealthDat}\OperatorTok{$}\NormalTok{Age}\OperatorTok{^}\DecValTok{2}
\NormalTok{HealthDat}\OperatorTok{$}\NormalTok{IndSq=HealthDat}\OperatorTok{$}\NormalTok{Ind}\OperatorTok{^}\DecValTok{2}


\NormalTok{fluSq.glm=}\KeywordTok{glm}\NormalTok{(FShot}\OperatorTok{~}\NormalTok{Age}\OperatorTok{+}\NormalTok{Ind}\OperatorTok{+}\NormalTok{AgeSq}\OperatorTok{+}\NormalTok{IndSq,}\DataTypeTok{family=}\NormalTok{binomial,}\DataTypeTok{data=}\NormalTok{HealthDat)}
\KeywordTok{summary}\NormalTok{(flutwo.glm)}
\end{Highlighting}
\end{Shaded}

\begin{verbatim}
## 
## Call:
## glm(formula = FShot ~ Age + Ind, family = binomial, data = HealthDat)
## 
## Deviance Residuals: 
##     Min       1Q   Median       3Q      Max  
## -1.4479  -0.5708  -0.3390  -0.1629   2.8430  
## 
## Coefficients:
##             Estimate Std. Error z value Pr(>|z|)   
## (Intercept) -1.45778    2.91534  -0.500  0.61705   
## Age          0.07787    0.02970   2.622  0.00873 **
## Ind         -0.09547    0.03241  -2.946  0.00322 **
## ---
## Signif. codes:  0 '***' 0.001 '**' 0.01 '*' 0.05 '.' 0.1 ' ' 1
## 
## (Dispersion parameter for binomial family taken to be 1)
## 
##     Null deviance: 134.94  on 158  degrees of freedom
## Residual deviance: 105.80  on 156  degrees of freedom
## AIC: 111.8
## 
## Number of Fisher Scoring iterations: 6
\end{verbatim}

\begin{Shaded}
\begin{Highlighting}[]
\KeywordTok{summary}\NormalTok{(fluSq.glm)}
\end{Highlighting}
\end{Shaded}

\begin{verbatim}
## 
## Call:
## glm(formula = FShot ~ Age + Ind + AgeSq + IndSq, family = binomial, 
##     data = HealthDat)
## 
## Deviance Residuals: 
##     Min       1Q   Median       3Q      Max  
## -1.7110  -0.5723  -0.3318  -0.2017   2.5082  
## 
## Coefficients:
##              Estimate Std. Error z value Pr(>|z|)
## (Intercept)  0.214549  14.260478   0.015    0.988
## Age          0.229664   0.405241   0.567    0.571
## Ind         -0.351741   0.263869  -1.333    0.183
## AgeSq       -0.001119   0.003028  -0.369    0.712
## IndSq        0.002384   0.002365   1.008    0.313
## 
## (Dispersion parameter for binomial family taken to be 1)
## 
##     Null deviance: 134.94  on 158  degrees of freedom
## Residual deviance: 104.71  on 154  degrees of freedom
## AIC: 114.71
## 
## Number of Fisher Scoring iterations: 6
\end{verbatim}

Ho:β3=β4=0 Ha: Not Ho Dev(reduced)-Dev(full)=105.80-104.71=1.09
p=P(\(X^2_2\)\textgreater{}1.09)

\begin{Shaded}
\begin{Highlighting}[]
\KeywordTok{anova}\NormalTok{(flutwo.glm, fluSq.glm, }\DataTypeTok{test =} \StringTok{'LRT'}\NormalTok{)}
\end{Highlighting}
\end{Shaded}

\begin{verbatim}
## Analysis of Deviance Table
## 
## Model 1: FShot ~ Age + Ind
## Model 2: FShot ~ Age + Ind + AgeSq + IndSq
##   Resid. Df Resid. Dev Df Deviance Pr(>Chi)
## 1       156     105.80                     
## 2       154     104.71  2   1.0894     0.58
\end{verbatim}

\begin{Shaded}
\begin{Highlighting}[]
\DecValTok{1}\OperatorTok{-}\KeywordTok{pchisq}\NormalTok{(}\FloatTok{1.09}\NormalTok{,}\DecValTok{2}\NormalTok{)}
\end{Highlighting}
\end{Shaded}

\begin{verbatim}
## [1] 0.5798418
\end{verbatim}

As p=0.5798\textgreater{}Alpha=0.05, we fail to reject the null
hypothesis and thus we can drop all of the predictors that were left out
of the reduced model.

Add interaction terms and test for their significance.

\begin{Shaded}
\begin{Highlighting}[]
\NormalTok{fluInt.glm=}\KeywordTok{glm}\NormalTok{(FShot}\OperatorTok{~}\NormalTok{Age}\OperatorTok{+}\NormalTok{Ind}\OperatorTok{+}\NormalTok{Age}\OperatorTok{:}\NormalTok{Ind,}\DataTypeTok{family=}\NormalTok{binomial,}\DataTypeTok{data=}\NormalTok{HealthDat)}
\KeywordTok{summary}\NormalTok{(flutwo.glm)}
\end{Highlighting}
\end{Shaded}

\begin{verbatim}
## 
## Call:
## glm(formula = FShot ~ Age + Ind, family = binomial, data = HealthDat)
## 
## Deviance Residuals: 
##     Min       1Q   Median       3Q      Max  
## -1.4479  -0.5708  -0.3390  -0.1629   2.8430  
## 
## Coefficients:
##             Estimate Std. Error z value Pr(>|z|)   
## (Intercept) -1.45778    2.91534  -0.500  0.61705   
## Age          0.07787    0.02970   2.622  0.00873 **
## Ind         -0.09547    0.03241  -2.946  0.00322 **
## ---
## Signif. codes:  0 '***' 0.001 '**' 0.01 '*' 0.05 '.' 0.1 ' ' 1
## 
## (Dispersion parameter for binomial family taken to be 1)
## 
##     Null deviance: 134.94  on 158  degrees of freedom
## Residual deviance: 105.80  on 156  degrees of freedom
## AIC: 111.8
## 
## Number of Fisher Scoring iterations: 6
\end{verbatim}

\begin{Shaded}
\begin{Highlighting}[]
\KeywordTok{summary}\NormalTok{(fluInt.glm)}
\end{Highlighting}
\end{Shaded}

\begin{verbatim}
## 
## Call:
## glm(formula = FShot ~ Age + Ind + Age:Ind, family = binomial, 
##     data = HealthDat)
## 
## Deviance Residuals: 
##     Min       1Q   Median       3Q      Max  
## -1.4239  -0.5872  -0.3409  -0.1469   2.8827  
## 
## Coefficients:
##               Estimate Std. Error z value Pr(>|z|)
## (Intercept)  1.2378112 13.0773400   0.095    0.925
## Age          0.0377140  0.1917167   0.197    0.844
## Ind         -0.1455268  0.2397655  -0.607    0.544
## Age:Ind      0.0007486  0.0035400   0.211    0.833
## 
## (Dispersion parameter for binomial family taken to be 1)
## 
##     Null deviance: 134.94  on 158  degrees of freedom
## Residual deviance: 105.75  on 155  degrees of freedom
## AIC: 113.75
## 
## Number of Fisher Scoring iterations: 6
\end{verbatim}

Ho:β3=0 Ha: Not Ho Dev(reduced)-Dev(full)=105.75-104.71=1.04
p=P(\(X^2_1\)\textgreater{}1.04)

\begin{Shaded}
\begin{Highlighting}[]
\KeywordTok{anova}\NormalTok{(flutwo.glm, fluInt.glm, }\DataTypeTok{test =} \StringTok{'LRT'}\NormalTok{)}
\end{Highlighting}
\end{Shaded}

\begin{verbatim}
## Analysis of Deviance Table
## 
## Model 1: FShot ~ Age + Ind
## Model 2: FShot ~ Age + Ind + Age:Ind
##   Resid. Df Resid. Dev Df Deviance Pr(>Chi)
## 1       156     105.80                     
## 2       155     105.75  1 0.045041   0.8319
\end{verbatim}

\begin{Shaded}
\begin{Highlighting}[]
\DecValTok{1}\OperatorTok{-}\KeywordTok{pchisq}\NormalTok{(}\FloatTok{1.04}\NormalTok{,}\DecValTok{1}\NormalTok{)}
\end{Highlighting}
\end{Shaded}

\begin{verbatim}
## [1] 0.3078215
\end{verbatim}

As p=0.3078\textgreater{}Alpha=0.05, we fail to reject the null
hypothesis and thus we can drop all of the predictors that were left out
of the reduced model.

Thus, our final best model includes no polynomial or interaction terms
and is log(odds of disease)=-1.458+0.079(Age)-0.095(Ind) \pagebreak

\section{Question 2.}\label{question-2.}

A disease outbreak has occurred in a certain city. Data have been
collected on a random telephone survey of 196 people within city limits
and the following data recorded: 1) Whether or not they have contracted
the disease (Dis, =1 if they have, =0 if not), Age, Socioeconomic Status
(SES, = 1 if upper, = 2 if middle, = 3 if lower), Sector of the city
they live (Sect, either sector 1 or sector 2), and saving account status
(Sav, = 1 if they have a savings account, = 0 if not). data appear in
Hwk6Q2DatSp19.

Part A) Develop a logistic regression model for predicting the
probability of contracting this disease, using the above variables. Be
sure to check for polynomial effects of significant quantitative
variables as well as interactions between significant predictor
variables. When finished, explicitly state your prediction equation. Be
sure to show significant steps in model development, using simultaneous
tests when you want to omit/test more than one predictor. Start by
factoring our binomial parameters and then obtaining a summary of the
model with all predictors in.

\begin{Shaded}
\begin{Highlighting}[]
\KeywordTok{library}\NormalTok{(readxl)}
\NormalTok{InfDat <-}\StringTok{ }\KeywordTok{read_excel}\NormalTok{(}\StringTok{"Hwk6Q2DatSp19.xlsx"}\NormalTok{)}
\NormalTok{InfDat}\OperatorTok{$}\NormalTok{Dis=}\KeywordTok{factor}\NormalTok{(InfDat}\OperatorTok{$}\NormalTok{Dis)}
\NormalTok{InfDat}\OperatorTok{$}\NormalTok{SES=}\KeywordTok{factor}\NormalTok{(InfDat}\OperatorTok{$}\NormalTok{SES)}
\NormalTok{InfDat}\OperatorTok{$}\NormalTok{Sect=}\KeywordTok{factor}\NormalTok{(InfDat}\OperatorTok{$}\NormalTok{Sect)}
\NormalTok{InfDat}\OperatorTok{$}\NormalTok{Sav=}\KeywordTok{factor}\NormalTok{(InfDat}\OperatorTok{$}\NormalTok{Sav)}

\NormalTok{inf.glm=}\KeywordTok{glm}\NormalTok{(Dis}\OperatorTok{~}\NormalTok{Age}\OperatorTok{+}\NormalTok{SES}\OperatorTok{+}\NormalTok{Sect}\OperatorTok{+}\NormalTok{Sav,}\DataTypeTok{family=}\NormalTok{binomial,}\DataTypeTok{data=}\NormalTok{InfDat)}

\KeywordTok{summary}\NormalTok{(inf.glm)}
\end{Highlighting}
\end{Shaded}

\begin{verbatim}
## 
## Call:
## glm(formula = Dis ~ Age + SES + Sect + Sav, family = binomial, 
##     data = InfDat)
## 
## Deviance Residuals: 
##     Min       1Q   Median       3Q      Max  
## -1.6614  -0.8309  -0.5630   1.0134   2.0918  
## 
## Coefficients:
##              Estimate Std. Error z value Pr(>|z|)    
## (Intercept) -2.273558   0.479333  -4.743  2.1e-06 ***
## Age          0.027280   0.009132   2.987 0.002813 ** 
## SES2         0.035578   0.441452   0.081 0.935765    
## SES3         0.237633   0.433750   0.548 0.583789    
## Sect2        1.249464   0.357009   3.500 0.000466 ***
## Sav1        -0.040692   0.396540  -0.103 0.918266    
## ---
## Signif. codes:  0 '***' 0.001 '**' 0.01 '*' 0.05 '.' 0.1 ' ' 1
## 
## (Dispersion parameter for binomial family taken to be 1)
## 
##     Null deviance: 236.33  on 195  degrees of freedom
## Residual deviance: 211.21  on 190  degrees of freedom
## AIC: 223.21
## 
## Number of Fisher Scoring iterations: 4
\end{verbatim}

SES and Sav appear to be insignificant so we will perform a simultaneous
test to see if thy can be dropped.

\begin{Shaded}
\begin{Highlighting}[]
\NormalTok{infred.glm=}\KeywordTok{glm}\NormalTok{(Dis}\OperatorTok{~}\NormalTok{Age}\OperatorTok{+}\NormalTok{Sect,}\DataTypeTok{family=}\NormalTok{binomial,}\DataTypeTok{data=}\NormalTok{InfDat)}
\KeywordTok{summary}\NormalTok{(infred.glm)}
\end{Highlighting}
\end{Shaded}

\begin{verbatim}
## 
## Call:
## glm(formula = Dis ~ Age + Sect, family = binomial, data = InfDat)
## 
## Deviance Residuals: 
##     Min       1Q   Median       3Q      Max  
## -1.6839  -0.8199  -0.5606   1.0093   2.0275  
## 
## Coefficients:
##             Estimate Std. Error z value Pr(>|z|)    
## (Intercept) -2.15966    0.34388  -6.280 3.38e-10 ***
## Age          0.02681    0.00865   3.100 0.001936 ** 
## Sect2        1.18169    0.33696   3.507 0.000453 ***
## ---
## Signif. codes:  0 '***' 0.001 '**' 0.01 '*' 0.05 '.' 0.1 ' ' 1
## 
## (Dispersion parameter for binomial family taken to be 1)
## 
##     Null deviance: 236.33  on 195  degrees of freedom
## Residual deviance: 211.64  on 193  degrees of freedom
## AIC: 217.64
## 
## Number of Fisher Scoring iterations: 3
\end{verbatim}

\begin{Shaded}
\begin{Highlighting}[]
\KeywordTok{anova}\NormalTok{(infred.glm, inf.glm, }\DataTypeTok{test =} \StringTok{'LRT'}\NormalTok{)}
\end{Highlighting}
\end{Shaded}

\begin{verbatim}
## Analysis of Deviance Table
## 
## Model 1: Dis ~ Age + Sect
## Model 2: Dis ~ Age + SES + Sect + Sav
##   Resid. Df Resid. Dev Df Deviance Pr(>Chi)
## 1       193     211.64                     
## 2       190     211.21  3  0.42984    0.934
\end{verbatim}

Ho: β2=β3=β5=0 Ha: Not Ho Dev(reduced)-Dev(full)=211.64-211.21=0.43
p=P(\(X^2_3\)\textgreater{}0.4)

\begin{Shaded}
\begin{Highlighting}[]
\DecValTok{1}\OperatorTok{-}\KeywordTok{pchisq}\NormalTok{(}\FloatTok{0.43}\NormalTok{,}\DecValTok{3}\NormalTok{)}
\end{Highlighting}
\end{Shaded}

\begin{verbatim}
## [1] 0.9339778
\end{verbatim}

As p=0.9339\textgreater{}Alpha=0.05, we fail to reject the null
hypothesis and thus we can drop all of the predictors that were left out
of the reduced model.

Next we will check to see if any polynomial terms are significant

\begin{Shaded}
\begin{Highlighting}[]
\NormalTok{InfDat}\OperatorTok{$}\NormalTok{AgeSq=InfDat}\OperatorTok{$}\NormalTok{Age}\OperatorTok{^}\DecValTok{2}
\NormalTok{InfDat}\OperatorTok{$}\NormalTok{AgeCub=InfDat}\OperatorTok{$}\NormalTok{Age}\OperatorTok{^}\DecValTok{3}

\NormalTok{infCub.glm=}\KeywordTok{glm}\NormalTok{(Dis}\OperatorTok{~}\NormalTok{Age}\OperatorTok{+}\NormalTok{Sect}\OperatorTok{+}\NormalTok{AgeSq}\OperatorTok{+}\NormalTok{AgeCub,}\DataTypeTok{family=}\NormalTok{binomial,}\DataTypeTok{data=}\NormalTok{InfDat)}
\NormalTok{infSq.glm=}\KeywordTok{glm}\NormalTok{(Dis}\OperatorTok{~}\NormalTok{Age}\OperatorTok{+}\NormalTok{Sect}\OperatorTok{+}\NormalTok{AgeSq,}\DataTypeTok{family=}\NormalTok{binomial,}\DataTypeTok{data=}\NormalTok{InfDat)}
\KeywordTok{summary}\NormalTok{(infCub.glm)}
\end{Highlighting}
\end{Shaded}

\begin{verbatim}
## 
## Call:
## glm(formula = Dis ~ Age + Sect + AgeSq + AgeCub, family = binomial, 
##     data = InfDat)
## 
## Deviance Residuals: 
##     Min       1Q   Median       3Q      Max  
## -1.4134  -0.8308  -0.4869   0.9689   2.2511  
## 
## Coefficients:
##               Estimate Std. Error z value Pr(>|z|)    
## (Intercept) -3.939e+00  8.642e-01  -4.559 5.15e-06 ***
## Age          1.990e-01  8.387e-02   2.373 0.017641 *  
## Sect2        1.311e+00  3.508e-01   3.738 0.000186 ***
## AgeSq       -3.955e-03  2.449e-03  -1.615 0.106407    
## AgeCub       2.395e-05  2.053e-05   1.167 0.243324    
## ---
## Signif. codes:  0 '***' 0.001 '**' 0.01 '*' 0.05 '.' 0.1 ' ' 1
## 
## (Dispersion parameter for binomial family taken to be 1)
## 
##     Null deviance: 236.33  on 195  degrees of freedom
## Residual deviance: 202.25  on 191  degrees of freedom
## AIC: 212.25
## 
## Number of Fisher Scoring iterations: 5
\end{verbatim}

\begin{Shaded}
\begin{Highlighting}[]
\KeywordTok{summary}\NormalTok{(infSq.glm)}
\end{Highlighting}
\end{Shaded}

\begin{verbatim}
## 
## Call:
## glm(formula = Dis ~ Age + Sect + AgeSq, family = binomial, data = InfDat)
## 
## Deviance Residuals: 
##     Min       1Q   Median       3Q      Max  
## -1.4809  -0.8260  -0.5211   0.9359   2.2128  
## 
## Coefficients:
##              Estimate Std. Error z value Pr(>|z|)    
## (Intercept) -3.281981   0.582123  -5.638 1.72e-08 ***
## Age          0.113425   0.033833   3.352 0.000801 ***
## Sect2        1.278602   0.348733   3.666 0.000246 ***
## AgeSq       -0.001194   0.000446  -2.677 0.007427 ** 
## ---
## Signif. codes:  0 '***' 0.001 '**' 0.01 '*' 0.05 '.' 0.1 ' ' 1
## 
## (Dispersion parameter for binomial family taken to be 1)
## 
##     Null deviance: 236.33  on 195  degrees of freedom
## Residual deviance: 203.58  on 192  degrees of freedom
## AIC: 211.58
## 
## Number of Fisher Scoring iterations: 4
\end{verbatim}

As we drop polynomial terms we drop until al terms are signifcant which
occurs when we have Age, Sect, and AgeSq(uared)

Next we will check to see if any interaction terms are signifcant.

\begin{Shaded}
\begin{Highlighting}[]
\NormalTok{infint.glm=}\KeywordTok{glm}\NormalTok{(Dis}\OperatorTok{~}\NormalTok{Age}\OperatorTok{+}\NormalTok{Sect}\OperatorTok{+}\NormalTok{AgeSq}\OperatorTok{+}\NormalTok{Age}\OperatorTok{:}\NormalTok{AgeSq}\OperatorTok{+}\NormalTok{Age}\OperatorTok{:}\NormalTok{Sect}\OperatorTok{+}\NormalTok{AgeSq}\OperatorTok{:}\NormalTok{Sect,}\DataTypeTok{family=}\NormalTok{binomial,}\DataTypeTok{data=}\NormalTok{InfDat)}
\KeywordTok{summary}\NormalTok{(infint.glm)}
\end{Highlighting}
\end{Shaded}

\begin{verbatim}
## 
## Call:
## glm(formula = Dis ~ Age + Sect + AgeSq + Age:AgeSq + Age:Sect + 
##     AgeSq:Sect, family = binomial, data = InfDat)
## 
## Deviance Residuals: 
##     Min       1Q   Median       3Q      Max  
## -1.7121  -0.8514  -0.5010   1.0300   2.3117  
## 
## Coefficients:
##               Estimate Std. Error z value Pr(>|z|)    
## (Intercept) -4.431e+00  1.165e+00  -3.804 0.000142 ***
## Age          2.448e-01  1.008e-01   2.428 0.015165 *  
## Sect2        2.087e+00  1.192e+00   1.750 0.080040 .  
## AgeSq       -4.833e-03  2.778e-03  -1.740 0.081910 .  
## Age:AgeSq    2.627e-05  2.340e-05   1.123 0.261454    
## Age:Sect2   -7.143e-02  7.406e-02  -0.964 0.334807    
## Sect2:AgeSq  1.131e-03  9.853e-04   1.148 0.250846    
## ---
## Signif. codes:  0 '***' 0.001 '**' 0.01 '*' 0.05 '.' 0.1 ' ' 1
## 
## (Dispersion parameter for binomial family taken to be 1)
## 
##     Null deviance: 236.33  on 195  degrees of freedom
## Residual deviance: 200.47  on 189  degrees of freedom
## AIC: 214.47
## 
## Number of Fisher Scoring iterations: 5
\end{verbatim}

\begin{Shaded}
\begin{Highlighting}[]
\KeywordTok{summary}\NormalTok{(infSq.glm)}
\end{Highlighting}
\end{Shaded}

\begin{verbatim}
## 
## Call:
## glm(formula = Dis ~ Age + Sect + AgeSq, family = binomial, data = InfDat)
## 
## Deviance Residuals: 
##     Min       1Q   Median       3Q      Max  
## -1.4809  -0.8260  -0.5211   0.9359   2.2128  
## 
## Coefficients:
##              Estimate Std. Error z value Pr(>|z|)    
## (Intercept) -3.281981   0.582123  -5.638 1.72e-08 ***
## Age          0.113425   0.033833   3.352 0.000801 ***
## Sect2        1.278602   0.348733   3.666 0.000246 ***
## AgeSq       -0.001194   0.000446  -2.677 0.007427 ** 
## ---
## Signif. codes:  0 '***' 0.001 '**' 0.01 '*' 0.05 '.' 0.1 ' ' 1
## 
## (Dispersion parameter for binomial family taken to be 1)
## 
##     Null deviance: 236.33  on 195  degrees of freedom
## Residual deviance: 203.58  on 192  degrees of freedom
## AIC: 211.58
## 
## Number of Fisher Scoring iterations: 4
\end{verbatim}

\begin{Shaded}
\begin{Highlighting}[]
\KeywordTok{anova}\NormalTok{(infSq.glm, infint.glm, }\DataTypeTok{test =} \StringTok{'LRT'}\NormalTok{)}
\end{Highlighting}
\end{Shaded}

\begin{verbatim}
## Analysis of Deviance Table
## 
## Model 1: Dis ~ Age + Sect + AgeSq
## Model 2: Dis ~ Age + Sect + AgeSq + Age:AgeSq + Age:Sect + AgeSq:Sect
##   Resid. Df Resid. Dev Df Deviance Pr(>Chi)
## 1       192     203.57                     
## 2       189     200.47  3   3.1047   0.3758
\end{verbatim}

Ho:β4=β5=β6=0 Ha: Not Ho Dev(reduced)-Dev(full)=203.58-200.47=3.11
p=P(\(X^2_3\)\textgreater{}3.11)

\begin{Shaded}
\begin{Highlighting}[]
\DecValTok{1}\OperatorTok{-}\KeywordTok{pchisq}\NormalTok{(}\FloatTok{3.11}\NormalTok{,}\DecValTok{3}\NormalTok{)}
\end{Highlighting}
\end{Shaded}

\begin{verbatim}
## [1] 0.3749743
\end{verbatim}

As p=0.375\textgreater{}Alpha=0.05, we fail to reject the null
hypothesis and thus we can drop all of the predictors that were left out
of the reduced model.

The final model uses predictors Age Sect and \((Age)^2\)

\begin{Shaded}
\begin{Highlighting}[]
\KeywordTok{summary}\NormalTok{(infSq.glm)}
\end{Highlighting}
\end{Shaded}

\begin{verbatim}
## 
## Call:
## glm(formula = Dis ~ Age + Sect + AgeSq, family = binomial, data = InfDat)
## 
## Deviance Residuals: 
##     Min       1Q   Median       3Q      Max  
## -1.4809  -0.8260  -0.5211   0.9359   2.2128  
## 
## Coefficients:
##              Estimate Std. Error z value Pr(>|z|)    
## (Intercept) -3.281981   0.582123  -5.638 1.72e-08 ***
## Age          0.113425   0.033833   3.352 0.000801 ***
## Sect2        1.278602   0.348733   3.666 0.000246 ***
## AgeSq       -0.001194   0.000446  -2.677 0.007427 ** 
## ---
## Signif. codes:  0 '***' 0.001 '**' 0.01 '*' 0.05 '.' 0.1 ' ' 1
## 
## (Dispersion parameter for binomial family taken to be 1)
## 
##     Null deviance: 236.33  on 195  degrees of freedom
## Residual deviance: 203.58  on 192  degrees of freedom
## AIC: 211.58
## 
## Number of Fisher Scoring iterations: 4
\end{verbatim}

The final prediction equation is log(odds of
infection)=-3.281981+0.113425(Age)+1.278602(Sect2)-0.001194(AgeSq).

Part B) Give a 90\% confidence interval for the probability that a 64
year old patient, with middle socioeconomic status and a savings account
that lives in sector 2 of the city, contracts the disease.

\begin{Shaded}
\begin{Highlighting}[]
\NormalTok{newdata <-}\StringTok{ }\KeywordTok{data.frame}\NormalTok{(}\DataTypeTok{Age=}\DecValTok{64}\NormalTok{, }\DataTypeTok{Sect=}\StringTok{"2"}\NormalTok{, }\DataTypeTok{AgeSq=}\DecValTok{4096}\NormalTok{)}
\KeywordTok{predict}\NormalTok{(infSq.glm, newdata,}\DataTypeTok{family=}\NormalTok{binomial, }\DataTypeTok{se.fit=}\OtherTok{TRUE}\NormalTok{,  }\DataTypeTok{interval=}\StringTok{"prediction"}\NormalTok{)}
\end{Highlighting}
\end{Shaded}

\begin{verbatim}
## $fit
##       1 
## 0.36516 
## 
## $se.fit
## [1] 0.4239494
## 
## $residual.scale
## [1] 1
\end{verbatim}

\begin{Shaded}
\begin{Highlighting}[]
\KeywordTok{summary}\NormalTok{(infSq.glm)}
\end{Highlighting}
\end{Shaded}

\begin{verbatim}
## 
## Call:
## glm(formula = Dis ~ Age + Sect + AgeSq, family = binomial, data = InfDat)
## 
## Deviance Residuals: 
##     Min       1Q   Median       3Q      Max  
## -1.4809  -0.8260  -0.5211   0.9359   2.2128  
## 
## Coefficients:
##              Estimate Std. Error z value Pr(>|z|)    
## (Intercept) -3.281981   0.582123  -5.638 1.72e-08 ***
## Age          0.113425   0.033833   3.352 0.000801 ***
## Sect2        1.278602   0.348733   3.666 0.000246 ***
## AgeSq       -0.001194   0.000446  -2.677 0.007427 ** 
## ---
## Signif. codes:  0 '***' 0.001 '**' 0.01 '*' 0.05 '.' 0.1 ' ' 1
## 
## (Dispersion parameter for binomial family taken to be 1)
## 
##     Null deviance: 236.33  on 195  degrees of freedom
## Residual deviance: 203.58  on 192  degrees of freedom
## AIC: 211.58
## 
## Number of Fisher Scoring iterations: 4
\end{verbatim}

\begin{Shaded}
\begin{Highlighting}[]
\KeywordTok{exp}\NormalTok{(}\FloatTok{0.36516}\NormalTok{)}
\end{Highlighting}
\end{Shaded}

\begin{verbatim}
## [1] 1.440745
\end{verbatim}

\begin{Shaded}
\begin{Highlighting}[]
\KeywordTok{qnorm}\NormalTok{(}\FloatTok{0.05}\NormalTok{)}
\end{Highlighting}
\end{Shaded}

\begin{verbatim}
## [1] -1.644854
\end{verbatim}

log(odds of inf)=-3.281981+0.113425(64)+1.278602(1)-0.001194(4096)

\begin{Shaded}
\begin{Highlighting}[]
\OperatorTok{-}\FloatTok{3.281981}\OperatorTok{+}\FloatTok{0.113425}\OperatorTok{*}\NormalTok{(}\DecValTok{64}\NormalTok{)}\OperatorTok{+}\FloatTok{1.278602}\OperatorTok{*}\NormalTok{(}\DecValTok{1}\NormalTok{)}\OperatorTok{-}\FloatTok{0.001194}\OperatorTok{*}\NormalTok{(}\DecValTok{4096}\NormalTok{)}
\end{Highlighting}
\end{Shaded}

\begin{verbatim}
## [1] 0.365197
\end{verbatim}

\begin{Shaded}
\begin{Highlighting}[]
\KeywordTok{exp}\NormalTok{(}\FloatTok{0.365197}\NormalTok{)}
\end{Highlighting}
\end{Shaded}

\begin{verbatim}
## [1] 1.440798
\end{verbatim}

We can be 90\% confident that odds of infection are between 0.365197 and
1.440798 for this man. Thus, p/1-p=0.365197 and p/1-p=1.440798. Thus, p
is between 0.2675 and 0.5903. Thus, we can be 90\% confident that the
probability of this man being infected with this disease is between
0.2675 and 0.5903. \#Question 3.

A new organic pesticide is being tested for effectiveness and required
dosage. An experiment was run where 30 randomly selected insects were
exposed to an environment treated with six different dosages of the
pesticide. Data observed were (in Hwk6Q3DatSp19):

Dead = Number out of 30 killed by the pesticide at the given dose Alive
= Number out of 30 not killed by the pesticide at the given dose Conc =
Log base 10 of the concentration (parts per trillion) (YES, THAT'S LOG
BASE 10!)

Research question: How do the odds of killing an insect change when the
concentration is increased by a factor of 10?

Your answer to this question should contain the following parts:

\begin{enumerate}
\def\labelenumi{\Alph{enumi})}
\tightlist
\item
  Formulation of the research question and choice of the appropriate
  statistical technique used to answer this question.
\end{enumerate}

The research question is how do the odds of killing an insect change
when the concentration is increased by a factor of 10. In studying the
question, we fit a model using binary logistic regression with dependent
variable being the ln(odds of killing an insect) and independent
variables being the log(concentraction of dosage). With the fitted
model, we can use a 95\% confidence interval to determine a confidence
range on the coefficient for log(concentraction of dosage).

\begin{enumerate}
\def\labelenumi{\Alph{enumi})}
\setcounter{enumi}{1}
\tightlist
\item
  Notation for the random variable(s) and parameter(s) of interest;
  define these explicitly. Give the distributional assumptions for your
  random variable(s) and state all assumptions necessary for the
  statistical application you intend to use.
\end{enumerate}

\begin{Shaded}
\begin{Highlighting}[]
\KeywordTok{getwd}\NormalTok{()}
\end{Highlighting}
\end{Shaded}

\begin{verbatim}
## [1] "/Users/kevinklaben/Downloads"
\end{verbatim}

\begin{Shaded}
\begin{Highlighting}[]
\KeywordTok{library}\NormalTok{(readxl)}
\NormalTok{KillDat <-}\StringTok{ }\KeywordTok{read_excel}\NormalTok{(}\StringTok{"Hwk6Q3DatSp19.xlsx"}\NormalTok{)}


\NormalTok{bug.glm =}\StringTok{ }\KeywordTok{glm}\NormalTok{(Dead}\OperatorTok{/}\NormalTok{(Dead}\OperatorTok{+}\NormalTok{Alive)}\OperatorTok{~}\NormalTok{Conc, }\DataTypeTok{family=} \StringTok{"quasibinomial"}\NormalTok{, }\DataTypeTok{weights=}\NormalTok{(Dead}\OperatorTok{+}\NormalTok{Alive),}\DataTypeTok{data=}\NormalTok{KillDat)}
\KeywordTok{summary}\NormalTok{(bug.glm)}
\end{Highlighting}
\end{Shaded}

\begin{verbatim}
## 
## Call:
## glm(formula = Dead/(Dead + Alive) ~ Conc, family = "quasibinomial", 
##     data = KillDat, weights = (Dead + Alive))
## 
## Deviance Residuals: 
##       1        2        3        4        5  
## -0.4510   0.3597   0.0000   0.0643  -0.2045  
## 
## Coefficients:
##             Estimate Std. Error t value Pr(>|t|)    
## (Intercept) -2.32379    0.14621  -15.89 0.000542 ***
## Conc         1.16189    0.06348   18.30 0.000356 ***
## ---
## Signif. codes:  0 '***' 0.001 '**' 0.01 '*' 0.05 '.' 0.1 ' ' 1
## 
## (Dispersion parameter for quasibinomial family taken to be 0.1224225)
## 
##     Null deviance: 64.76327  on 4  degrees of freedom
## Residual deviance:  0.37875  on 3  degrees of freedom
## AIC: NA
## 
## Number of Fisher Scoring iterations: 4
\end{verbatim}

The binomial logistic regression is used with the following model:
Yi=ln(pi(Xi)/(1-pi(Xi)))= β0 + β1Xi We make the following assumptions
for the model: Xi=the log of the concentration of pesticide used Yi=the
probability that an insect dies for a given log of concentration of
pesticide. β0=the ln(odds of the insect dying) when the
log(concentration)=0. β1=ln(odds ratio when log(concentration) is
increase by 1) b.i) The observations are independent b.ii) The data are
assumed to be distributed according to the binomial distribution with
the probability of success being pi(x)=1/(1+e\^{}-(β0 + β1Xi)).

\begin{enumerate}
\def\labelenumi{\Alph{enumi})}
\setcounter{enumi}{2}
\tightlist
\item
  Calculations for the analysis. For hypothesis and significance tests,
  formulate the null and the alternative hypotheses, calculate the value
  of your test statistic, and then calculate your p-value. For
  confidence intervals, show and apply the appropriate formula. Use
  \(\alpha\) = .05 if not otherwise specified.
\end{enumerate}

Perform a Wald test to confirm that the coefficient β1 is significant
Ho: β1=0 Ha: β1\textgreater{}0

\begin{Shaded}
\begin{Highlighting}[]
\NormalTok{(}\FloatTok{1.16189}\OperatorTok{-}\DecValTok{0}\NormalTok{)}\OperatorTok{/}\FloatTok{0.06348}
\end{Highlighting}
\end{Shaded}

\begin{verbatim}
## [1] 18.30325
\end{verbatim}

\begin{Shaded}
\begin{Highlighting}[]
\FloatTok{0.000356}\OperatorTok{/}\DecValTok{2}
\end{Highlighting}
\end{Shaded}

\begin{verbatim}
## [1] 0.000178
\end{verbatim}

z=Test statistic=((pt est)-(hyp
value))/SE(est)=(1.16189-0)/0.06348=18.30325
p(18.30325\textgreater{}0)=0.000356/2=0.000178 As
p=0.000178\textless{}alpha=0.05, we reject the null hypothesis and
conclude that the coefficient is significant and shouold be included as
part of the model as a predictor.

We will not compute a 95\% confidence interval for β1.

z(alpha/2)=1.96

\begin{Shaded}
\begin{Highlighting}[]
\FloatTok{1.16189}\OperatorTok{-}\FloatTok{1.96}\OperatorTok{*}\NormalTok{(}\FloatTok{0.06348}\NormalTok{)}
\end{Highlighting}
\end{Shaded}

\begin{verbatim}
## [1] 1.037469
\end{verbatim}

\begin{Shaded}
\begin{Highlighting}[]
\FloatTok{1.16189}\OperatorTok{+}\FloatTok{1.96}\OperatorTok{*}\NormalTok{(}\FloatTok{0.06348}\NormalTok{)}
\end{Highlighting}
\end{Shaded}

\begin{verbatim}
## [1] 1.286311
\end{verbatim}

1.16189 +/-1.96(0.06348)=(1.037469,1.286311) these must be exponetiated
with base e as

\begin{Shaded}
\begin{Highlighting}[]
\KeywordTok{exp}\NormalTok{(}\FloatTok{1.037469}\NormalTok{)}
\end{Highlighting}
\end{Shaded}

\begin{verbatim}
## [1] 2.822065
\end{verbatim}

\begin{Shaded}
\begin{Highlighting}[]
\KeywordTok{exp}\NormalTok{(}\FloatTok{1.286311}\NormalTok{)}
\end{Highlighting}
\end{Shaded}

\begin{verbatim}
## [1] 3.61941
\end{verbatim}

Thus, we can be 95\% confident that the odds of killing an insect go up
by between 182.2\% and 261.9\% for each increase in concentration by a
factor of 10.

\begin{enumerate}
\def\labelenumi{\Alph{enumi})}
\setcounter{enumi}{3}
\tightlist
\item
  Discuss whether the assumptions stated in Part B above are met
  sufficiently for the validity of the statistical inferences; use
  graphs and other tools where applicable.
\end{enumerate}

As it is stated that the insect were selected randomly, we can assume
that the insects are independent.

To test whether the assumption that pi(x) follows a logistic model we
can perfomr a goodness of fit test. Ho: β1=0 Ha: Not Ho

\begin{Shaded}
\begin{Highlighting}[]
\KeywordTok{summary}\NormalTok{(bug.glm)}
\end{Highlighting}
\end{Shaded}

\begin{verbatim}
## 
## Call:
## glm(formula = Dead/(Dead + Alive) ~ Conc, family = "quasibinomial", 
##     data = KillDat, weights = (Dead + Alive))
## 
## Deviance Residuals: 
##       1        2        3        4        5  
## -0.4510   0.3597   0.0000   0.0643  -0.2045  
## 
## Coefficients:
##             Estimate Std. Error t value Pr(>|t|)    
## (Intercept) -2.32379    0.14621  -15.89 0.000542 ***
## Conc         1.16189    0.06348   18.30 0.000356 ***
## ---
## Signif. codes:  0 '***' 0.001 '**' 0.01 '*' 0.05 '.' 0.1 ' ' 1
## 
## (Dispersion parameter for quasibinomial family taken to be 0.1224225)
## 
##     Null deviance: 64.76327  on 4  degrees of freedom
## Residual deviance:  0.37875  on 3  degrees of freedom
## AIC: NA
## 
## Number of Fisher Scoring iterations: 4
\end{verbatim}

\begin{Shaded}
\begin{Highlighting}[]
\NormalTok{bugnull.glm =}\StringTok{ }\KeywordTok{glm}\NormalTok{(Dead}\OperatorTok{/}\NormalTok{(Dead}\OperatorTok{+}\NormalTok{Alive)}\OperatorTok{~}\DecValTok{1}\NormalTok{, }\DataTypeTok{family=} \StringTok{"quasibinomial"}\NormalTok{, }\DataTypeTok{weights=}\NormalTok{(Dead}\OperatorTok{+}\NormalTok{Alive),}\DataTypeTok{data=}\NormalTok{KillDat)}
\KeywordTok{summary}\NormalTok{(bugnull.glm)}
\end{Highlighting}
\end{Shaded}

\begin{verbatim}
## 
## Call:
## glm(formula = Dead/(Dead + Alive) ~ 1, family = "quasibinomial", 
##     data = KillDat, weights = (Dead + Alive))
## 
## Deviance Residuals: 
##      1       2       3       4       5  
## -5.186  -2.607   0.000   2.999   4.699  
## 
## Coefficients:
##              Estimate Std. Error t value Pr(>|t|)
## (Intercept) 1.886e-15  6.154e-01       0        1
## 
## (Dispersion parameter for quasibinomial family taken to be 14.2)
## 
##     Null deviance: 64.763  on 4  degrees of freedom
## Residual deviance: 64.763  on 4  degrees of freedom
## AIC: NA
## 
## Number of Fisher Scoring iterations: 3
\end{verbatim}

\begin{Shaded}
\begin{Highlighting}[]
\FloatTok{64.763}\OperatorTok{-}\FloatTok{0.37875}
\end{Highlighting}
\end{Shaded}

\begin{verbatim}
## [1] 64.38425
\end{verbatim}

TS= Null deviance- residual deviance=64.763-0.37875= 64.38425 \(X^2\) df
is the number of observations-number of estimated parameters=5
concentrations-2 estmated parameters=3 df

\begin{Shaded}
\begin{Highlighting}[]
\DecValTok{1}\OperatorTok{-}\KeywordTok{pchisq}\NormalTok{(}\FloatTok{64.38425}\NormalTok{,}\DecValTok{3}\NormalTok{)}
\end{Highlighting}
\end{Shaded}

\begin{verbatim}
## [1] 6.794565e-14
\end{verbatim}

P(\$X\^{}2\_3\textgreater{}64.38425)= 6.794565e-14

As p=6.794565e-14\textless{}alpha=0.05, we reject the null hypothesis
and conclude that the coefficient is significnat and that pi(x) follows
a logistic model.

\begin{enumerate}
\def\labelenumi{\Alph{enumi})}
\setcounter{enumi}{4}
\tightlist
\item
  Discuss the sampling scheme and whether or not it is sufficient to
  meet the objective of the study. Be sure to include whether or not
  subjective inference is necessary and if so, defend whether or not you
  believe it is valid.
\end{enumerate}

The sampling scheme here is that insects are randomly selected and
treated by one of the five concentration levels. As we are interested in
the odds of insects dying when subjected to pesticide concentrations,
this sample is sufficient to meet the objective of the study. As the
sample population can be assumed to be selected from the overall insect
population randomly and population of interest is the insect population,
subjective inference is not necessary as the sample
population=population of interest.

\begin{enumerate}
\def\labelenumi{\Alph{enumi})}
\setcounter{enumi}{5}
\tightlist
\item
  State the conclusions of the analysis. These should be practical
  conclusions from the context of the problem, but should also be backed
  up with statistical criteria (like a p-value, etc.). Include any
  considerations such as limitations of the sampling scheme, impact of
  outliers, etc., that you feel must be considered when you state your
  conclusions.
\end{enumerate}

After analysis of the data by fitting with a binary logistic regression,
we can conclude that we are 95\% confident that one average for each
increase in concentration of pesticide by a factor of ten,the
probability of an insect dying can be expected to increase by between
182.2\% and 261.9\%. As our assumptions were satisfied and the sample
scheme is sufficient to meet the objective of the study we hold no
reservations about these conclusions.


\end{document}
