\documentclass[12pt]{article}
%\usepackage{fullpage}
\usepackage{epic}
\usepackage{eepic}
\usepackage{paralist}
\usepackage{graphicx}
\usepackage{algorithm,algorithmic}
\usepackage{tikz}
\usepackage{xcolor,colortbl}
\usepackage{wrapfig}


%%%%%%%%%%%%%%%%%%%%%%%%%%%%%%%%%%%%%%%%%%%%%%%%%%%%%%%%%%%%%%%%
% This is FULLPAGE.STY by H.Partl, Version 2 as of 15 Dec 1988.
% Document Style Option to fill the paper just like Plain TeX.

\typeout{Style Option FULLPAGE Version 2 as of 15 Dec 1988}

\topmargin 0pt
\advance \topmargin by -\headheight
\advance \topmargin by -\headsep

\textheight 8.9in

\oddsidemargin 0pt
\evensidemargin \oddsidemargin
\marginparwidth 0.5in

\textwidth 6.5in
%%%%%%%%%%%%%%%%%%%%%%%%%%%%%%%%%%%%%%%%%%%%%%%%%%%%%%%%%%%%%%%%

\pagestyle{empty}
\setlength{\oddsidemargin}{0in}
\setlength{\topmargin}{-0.8in}
\setlength{\textwidth}{6.8in}
\setlength{\textheight}{9.5in}


\def\ind{\hspace*{0.3in}}
\def\gap{0.1in}
\def\bigap{0.25in}
\newcommand{\Xomit}[1]{}


\begin{document}

\setlength{\parindent}{0in}
\addtolength{\parskip}{0.1cm}
\setlength{\fboxrule}{.5mm}\setlength{\fboxsep}{1.2mm}
\newlength{\boxlength}\setlength{\boxlength}{\textwidth}
\addtolength{\boxlength}{-4mm}
\begin{center}\framebox{\parbox{\boxlength}{{\bf
CS 4820, Spring 2019 \hfill Homework 3, Problem 1}\\
% TODO: fill in your own name, netID, and collaborators
Name: \\
NetID: \\
Collaborators:
}}
\end{center}
\vspace{5mm}
\vskip \bigap
{\bf (1)} {\em (10 points)}
In American politics, {\em gerrymandering} refers to the process of
subdividing a region into electoral districts to maximize a particular
political party's advantage. This problem explores a simplified model
of gerrymandering in which the region is modeled as one-dimensional.
Assume that there are $n > 0$ {\em precincts} represented by the
vertices $v_1,v_2,\ldots,v_n$ of an undirected path.
A {\em district} is defined to be a contiguous interval of precincts;
in other words a district is specified by its endpoints $i \le j$,
and it consists of precincts $v_i,v_{i+1},v_{i+2},\ldots,v_j$.
We will refer to such a district as $[i,j]$.

Assume there are two parties A and B competing in the election,
and for every  $ 1 \le i \le  j \le n$, $P[i,j]$ denotes the
probability that A wins in the district $[i,j]$.
The probability matrix $P$ is given as part of the input.
Assume that the law requires the precincts to be partitioned
into exactly $k$ disjoint districts, each containing at least
$s_{\min}$ and at most $s_{\max}$ nodes. You may also assume
the parameters $n,k,s_{\min},s_{\max}$ are chosen such that
there is at least one way to partition the precincts into
$k$ districts meeting the specified size constraints.
Your task is to find an efficient algorithm to gerrymander the
precincts into $k$ districts satisfying the size constraints,
so as to maximize the expected number of districts that A wins.

\vskip \bigap


%% Your solution goes here.

\end{document}
