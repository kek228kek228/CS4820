\documentclass[12pt]{article}
%\usepackage{fullpage}
\usepackage{epic}
\usepackage{eepic}
\usepackage{paralist}
\usepackage{graphicx}
\usepackage{algorithm,algorithmic}
\usepackage{tikz}
\usepackage{xcolor,colortbl}
\usepackage{wrapfig}


%%%%%%%%%%%%%%%%%%%%%%%%%%%%%%%%%%%%%%%%%%%%%%%%%%%%%%%%%%%%%%%%
% This is FULLPAGE.STY by H.Partl, Version 2 as of 15 Dec 1988.
% Document Style Option to fill the paper just like Plain TeX.

\typeout{Style Option FULLPAGE Version 2 as of 15 Dec 1988}

\topmargin 0pt
\advance \topmargin by -\headheight
\advance \topmargin by -\headsep

\textheight 8.9in

\oddsidemargin 0pt
\evensidemargin \oddsidemargin
\marginparwidth 0.5in

\textwidth 6.5in
%%%%%%%%%%%%%%%%%%%%%%%%%%%%%%%%%%%%%%%%%%%%%%%%%%%%%%%%%%%%%%%%

\pagestyle{empty}
\setlength{\oddsidemargin}{0in}
\setlength{\topmargin}{-0.8in}
\setlength{\textwidth}{6.8in}
\setlength{\textheight}{9.5in}


\def\ind{\hspace*{0.3in}}
\def\gap{0.1in}
\def\bigap{0.25in}
\newcommand{\Xomit}[1]{}


\begin{document}

\setlength{\parindent}{0in}
\addtolength{\parskip}{0.1cm}
\setlength{\fboxrule}{.5mm}\setlength{\fboxsep}{1.2mm}
\newlength{\boxlength}\setlength{\boxlength}{\textwidth}
\addtolength{\boxlength}{-4mm}
\begin{center}\framebox{\parbox{\boxlength}{{\bf
CS 4820, Spring 2019 \hfill Homework 2, Problem 2}\\
% TODO: fill in your own name, netID, and collaborators
Name: \\
NetID: \\
Collaborators:
}}
\end{center}
\vspace{5mm}

{\bf (2)} {\em (10 points)}
In the sport of American football, teams move a ball
toward the opposing team's goal line in a sequence
of plays called an offensive drive.
Subject to some simplifications, the game
has the following rules.
\begin{enumerate}
  \item An offensive drive starts $n$ yards away from
  the other team's goal line. In American football, $n$
  is often but not always equal to 80.
  \item To retain possession of the ball, the team
  must move forward at least $k$ yards in its first
  $d$ plays, called ``downs''. In American football,
  $k=10$ and $d=4$.
  \item More generally, certain plays in an offensive
  drive are called ``first downs''. The initial play in
  the drive is a first down. A subsequent play is
  called a first down if and only if the total
  yardage accumulated since the preceding first down
  is greater than or equal to $k$.
  \item There are three ways that a drive could end.
  \begin{itemize}
    \item If the total yardage accumulated in the drive
    is greater than or equal to $n$, the team scores a
    touchdown.
    \item If the team has completed $d$ plays
    since the last first down, and the total
    yardage accumulated in those $d$ plays
    is less than $k$, the team loses possession.
    \item Any play could result in a ``turnover''.
    If this happens, the team loses possession
    immediately.
  \end{itemize}
  \item On any given play of the drive, the team
  needs to decide whether to try a passing play
  or a running play. The probability of a turnover,
  and the distribution of the random number of
  yards gained in the event that the play is not
  a turnover, depend on whether the team chooses
  passing or running. We will assume that the
  number of yards gained is always a non-negative
  integer.\footnote{In the actual sport of football,
  it's possible for a play to gain a negative number
  of yards. We are deliberately ignoring this
  possibility, for the sake of simplicity.}
\end{enumerate}
A {\em football strategy} is a strategy
for choosing between a running or passing play
in every possible situation that may arise
during an offensive drive. A strategy is
{\em optimal} if it maximizes the probability
of scoring a touchdown.

For an integer $y$ in the range $0 \le y \le n$,
let $p_y$
denote the probability of gaining $y$ yards
on a passing play, and let $r_y$ denote the
probability of gaining $y$ yards on a running
play. Let $p_{n+1}$ and $r_{n+1}$ denote the
probability of a turnover on a passing or
running play, respectively.
Assume these are non-negative
rational numbers and that $\sum_{y=0}^{n+1} p_y
= \sum_{y=0}^{n+1} r_y = 1$.
Design an algorithm which is given the
parameters $n,k,d$ and the probabilities
$p_y, \, r_y$ for $y=0,1,\ldots,n+1$, and
which determines whether the optimal football strategy
chooses to run or to pass at the start of an
offensive drive, $n$ yards away from the
opponent's goal line. Your algorithm should work
for {\em any} values of $n,k,d$, not only for
$n=80, k=10, d=4$.


\vskip \bigap

%% Your solution goes here.

\end{document}
