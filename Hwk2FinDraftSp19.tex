\documentclass[12pt,]{article}
\usepackage{lmodern}
\usepackage{setspace}
\setstretch{1.2}
\usepackage{amssymb,amsmath}
\usepackage{ifxetex,ifluatex}
\usepackage{fixltx2e} % provides \textsubscript
\ifnum 0\ifxetex 1\fi\ifluatex 1\fi=0 % if pdftex
  \usepackage[T1]{fontenc}
  \usepackage[utf8]{inputenc}
\else % if luatex or xelatex
  \ifxetex
    \usepackage{mathspec}
  \else
    \usepackage{fontspec}
  \fi
  \defaultfontfeatures{Ligatures=TeX,Scale=MatchLowercase}
\fi
% use upquote if available, for straight quotes in verbatim environments
\IfFileExists{upquote.sty}{\usepackage{upquote}}{}
% use microtype if available
\IfFileExists{microtype.sty}{%
\usepackage{microtype}
\UseMicrotypeSet[protrusion]{basicmath} % disable protrusion for tt fonts
}{}
\usepackage[margin=1in]{geometry}
\usepackage{hyperref}
\hypersetup{unicode=true,
            pdftitle={BTRY 3020/STSCI 3200 Homework II},
            pdfborder={0 0 0},
            breaklinks=true}
\urlstyle{same}  % don't use monospace font for urls
\usepackage{color}
\usepackage{fancyvrb}
\newcommand{\VerbBar}{|}
\newcommand{\VERB}{\Verb[commandchars=\\\{\}]}
\DefineVerbatimEnvironment{Highlighting}{Verbatim}{commandchars=\\\{\}}
% Add ',fontsize=\small' for more characters per line
\usepackage{framed}
\definecolor{shadecolor}{RGB}{248,248,248}
\newenvironment{Shaded}{\begin{snugshade}}{\end{snugshade}}
\newcommand{\KeywordTok}[1]{\textcolor[rgb]{0.13,0.29,0.53}{\textbf{#1}}}
\newcommand{\DataTypeTok}[1]{\textcolor[rgb]{0.13,0.29,0.53}{#1}}
\newcommand{\DecValTok}[1]{\textcolor[rgb]{0.00,0.00,0.81}{#1}}
\newcommand{\BaseNTok}[1]{\textcolor[rgb]{0.00,0.00,0.81}{#1}}
\newcommand{\FloatTok}[1]{\textcolor[rgb]{0.00,0.00,0.81}{#1}}
\newcommand{\ConstantTok}[1]{\textcolor[rgb]{0.00,0.00,0.00}{#1}}
\newcommand{\CharTok}[1]{\textcolor[rgb]{0.31,0.60,0.02}{#1}}
\newcommand{\SpecialCharTok}[1]{\textcolor[rgb]{0.00,0.00,0.00}{#1}}
\newcommand{\StringTok}[1]{\textcolor[rgb]{0.31,0.60,0.02}{#1}}
\newcommand{\VerbatimStringTok}[1]{\textcolor[rgb]{0.31,0.60,0.02}{#1}}
\newcommand{\SpecialStringTok}[1]{\textcolor[rgb]{0.31,0.60,0.02}{#1}}
\newcommand{\ImportTok}[1]{#1}
\newcommand{\CommentTok}[1]{\textcolor[rgb]{0.56,0.35,0.01}{\textit{#1}}}
\newcommand{\DocumentationTok}[1]{\textcolor[rgb]{0.56,0.35,0.01}{\textbf{\textit{#1}}}}
\newcommand{\AnnotationTok}[1]{\textcolor[rgb]{0.56,0.35,0.01}{\textbf{\textit{#1}}}}
\newcommand{\CommentVarTok}[1]{\textcolor[rgb]{0.56,0.35,0.01}{\textbf{\textit{#1}}}}
\newcommand{\OtherTok}[1]{\textcolor[rgb]{0.56,0.35,0.01}{#1}}
\newcommand{\FunctionTok}[1]{\textcolor[rgb]{0.00,0.00,0.00}{#1}}
\newcommand{\VariableTok}[1]{\textcolor[rgb]{0.00,0.00,0.00}{#1}}
\newcommand{\ControlFlowTok}[1]{\textcolor[rgb]{0.13,0.29,0.53}{\textbf{#1}}}
\newcommand{\OperatorTok}[1]{\textcolor[rgb]{0.81,0.36,0.00}{\textbf{#1}}}
\newcommand{\BuiltInTok}[1]{#1}
\newcommand{\ExtensionTok}[1]{#1}
\newcommand{\PreprocessorTok}[1]{\textcolor[rgb]{0.56,0.35,0.01}{\textit{#1}}}
\newcommand{\AttributeTok}[1]{\textcolor[rgb]{0.77,0.63,0.00}{#1}}
\newcommand{\RegionMarkerTok}[1]{#1}
\newcommand{\InformationTok}[1]{\textcolor[rgb]{0.56,0.35,0.01}{\textbf{\textit{#1}}}}
\newcommand{\WarningTok}[1]{\textcolor[rgb]{0.56,0.35,0.01}{\textbf{\textit{#1}}}}
\newcommand{\AlertTok}[1]{\textcolor[rgb]{0.94,0.16,0.16}{#1}}
\newcommand{\ErrorTok}[1]{\textcolor[rgb]{0.64,0.00,0.00}{\textbf{#1}}}
\newcommand{\NormalTok}[1]{#1}
\usepackage{graphicx,grffile}
\makeatletter
\def\maxwidth{\ifdim\Gin@nat@width>\linewidth\linewidth\else\Gin@nat@width\fi}
\def\maxheight{\ifdim\Gin@nat@height>\textheight\textheight\else\Gin@nat@height\fi}
\makeatother
% Scale images if necessary, so that they will not overflow the page
% margins by default, and it is still possible to overwrite the defaults
% using explicit options in \includegraphics[width, height, ...]{}
\setkeys{Gin}{width=\maxwidth,height=\maxheight,keepaspectratio}
\IfFileExists{parskip.sty}{%
\usepackage{parskip}
}{% else
\setlength{\parindent}{0pt}
\setlength{\parskip}{6pt plus 2pt minus 1pt}
}
\setlength{\emergencystretch}{3em}  % prevent overfull lines
\providecommand{\tightlist}{%
  \setlength{\itemsep}{0pt}\setlength{\parskip}{0pt}}
\setcounter{secnumdepth}{0}
% Redefines (sub)paragraphs to behave more like sections
\ifx\paragraph\undefined\else
\let\oldparagraph\paragraph
\renewcommand{\paragraph}[1]{\oldparagraph{#1}\mbox{}}
\fi
\ifx\subparagraph\undefined\else
\let\oldsubparagraph\subparagraph
\renewcommand{\subparagraph}[1]{\oldsubparagraph{#1}\mbox{}}
\fi

%%% Use protect on footnotes to avoid problems with footnotes in titles
\let\rmarkdownfootnote\footnote%
\def\footnote{\protect\rmarkdownfootnote}

%%% Change title format to be more compact
\usepackage{titling}

% Create subtitle command for use in maketitle
\newcommand{\subtitle}[1]{
  \posttitle{
    \begin{center}\large#1\end{center}
    }
}

\setlength{\droptitle}{-2em}

  \title{BTRY 3020/STSCI 3200 Homework II}
    \pretitle{\vspace{\droptitle}\centering\huge}
  \posttitle{\par}
    \author{}
    \preauthor{}\postauthor{}
    \date{}
    \predate{}\postdate{}
  
\usepackage[utf8]{inputenc}
\usepackage[english]{babel}
\usepackage{amsmath}
\usepackage{amssymb}
\usepackage{amsthm}
\usepackage{booktabs}
\usepackage{longtable}
\usepackage{array}
\usepackage{multirow}
\usepackage{wrapfig}
\usepackage{float}
\usepackage{colortbl}
\usepackage{pdflscape}
\usepackage{tabu}
\usepackage{threeparttable}
\usepackage{threeparttablex}
\usepackage[normalem]{ulem}
\usepackage{makecell}
\usepackage{xcolor}

\begin{document}
\maketitle

\begin{center}\rule{0.5\linewidth}{\linethickness}\end{center}

\section{NAME: Kevin Klaben}\label{name-kevin-klaben}

\section{NETID: kek228}\label{netid-kek228}

\section{\texorpdfstring{\textbf{DUE DATE: February 19 2019, by
electronic submission of a PDF on BlackBoard by 8:40
am}}{DUE DATE: February 19 2019, by electronic submission of a PDF on BlackBoard by 8:40 am}}\label{due-date-february-19-2019-by-electronic-submission-of-a-pdf-on-blackboard-by-840-am}

\begin{center}\rule{0.5\linewidth}{\linethickness}\end{center}

\section{Question 1.}\label{question-1.}

An experiment was conducted where 13 subjects were asked to remember
disconnected items in a list. They were then randomly assigned a time
from 1 to 10080 minutes, each then waited this amount of time, and
subsequently were then asked about the items in the list. The proportion
of items (prop) correctly recalled after these various times (time, in
minutes) since the list was memorized were recorded (Hwk2Q1DatSp19).

\begin{enumerate}
\def\labelenumi{\alph{enumi})}
\tightlist
\item
  AS ALWAYS, plot the data and determine if linear regression is
  appropriate.
\end{enumerate}

\begin{Shaded}
\begin{Highlighting}[]
\KeywordTok{library}\NormalTok{(readxl)}
\KeywordTok{getwd}\NormalTok{()}
\end{Highlighting}
\end{Shaded}

\begin{verbatim}
## [1] "/Users/kevinklaben/Downloads"
\end{verbatim}

\begin{Shaded}
\begin{Highlighting}[]
\NormalTok{Pdata <-}\StringTok{ }\KeywordTok{read_excel}\NormalTok{(}\StringTok{"Hwk2Q1DatSp19.xlsx"}\NormalTok{)}
\KeywordTok{plot}\NormalTok{(Pdata}\OperatorTok{$}\NormalTok{time, Pdata}\OperatorTok{$}\NormalTok{prop, }\DataTypeTok{xlab =} \StringTok{"Time since last memorization in minutes"}\NormalTok{, }\DataTypeTok{ylab =} \StringTok{"Proportion of items remembered"}\NormalTok{, }\DataTypeTok{main =} \StringTok{"Time since last memorization of a list of disconnected items vs. Proposition of items remembered"}\NormalTok{)}
\end{Highlighting}
\end{Shaded}

\includegraphics{Hwk2FinDraftSp19_files/figure-latex/unnamed-chunk-1-1.pdf}
A linear regression of this raw data is clearly not applicable. There is
a clear curvature to the data thus it should not be fitted with a linear
regression. b) Regardless of the results from part a, go ahead and run
the regression and get the diagnostic plots. i) What does the residual
plot tell you?

\begin{Shaded}
\begin{Highlighting}[]
\KeywordTok{library}\NormalTok{(MASS)}
\NormalTok{Pdata.lm <-}\StringTok{ }\KeywordTok{lm}\NormalTok{(prop}\OperatorTok{~}\NormalTok{time, }\DataTypeTok{data =}\NormalTok{ Pdata)}
\NormalTok{Resids=}\KeywordTok{studres}\NormalTok{(Pdata.lm)}
\KeywordTok{plot}\NormalTok{(Pdata.lm}\OperatorTok{$}\NormalTok{fitted.values,Resids)}
\KeywordTok{abline}\NormalTok{(}\DataTypeTok{h=}\DecValTok{0}\NormalTok{)}
\end{Highlighting}
\end{Shaded}

\includegraphics{Hwk2FinDraftSp19_files/figure-latex/unnamed-chunk-2-1.pdf}
The residual plot tells us that this data is heteroskedastic. Thus, the
variance of the data across the x's is not constant and follows a
non-linear relationship. This informs us that a linear regression is not
applicable with the current state of the data.

\begin{verbatim}
ii) What does the qqPlot tell you?
\end{verbatim}

\begin{Shaded}
\begin{Highlighting}[]
\KeywordTok{library}\NormalTok{(car)}
\end{Highlighting}
\end{Shaded}

\begin{verbatim}
## Loading required package: carData
\end{verbatim}

\begin{Shaded}
\begin{Highlighting}[]
\KeywordTok{qqPlot}\NormalTok{(Pdata.lm}\OperatorTok{$}\NormalTok{residuals)}
\end{Highlighting}
\end{Shaded}

\includegraphics{Hwk2FinDraftSp19_files/figure-latex/unnamed-chunk-3-1.pdf}

\begin{verbatim}
## [1] 1 2
\end{verbatim}

The qqplot tells us that the data is normal interms of the y's at every
x. As all points fall within the 95\% confidence bands we can be fairly
certain that the proportions normally distributed at each time.

\begin{enumerate}
\def\labelenumi{\alph{enumi})}
\setcounter{enumi}{2}
\tightlist
\item
  Based on the plots from parts a and b above, suggest a transformation
  that you expect to give you correct diagnostic plots. DEFEND YOUR
  ANSWER. The transformation I suggest is taking the log of the
  predictor variable time. This makes sense as we need to bring the x's
  in this by taking the log of the x's, the large x values will get
  smaller by a large amount while the smaller x values will get only a
  small amount smaller. This should fix both the heterskedasticity and
  the non-linear relationship
\item
  Perform the transformation you suggested and plot the transformed
  data. Does it now appear that linear regression is appropriate on the
  transformed data? If not, go back and try another transformation and
  plot that to see if linear regression appears appropriate for this new
  tramnsformation. Repeat until you've found a transformation that
  linearizes the relationship.
\end{enumerate}

\begin{Shaded}
\begin{Highlighting}[]
\NormalTok{Pdata}\OperatorTok{$}\NormalTok{logtime<-}\KeywordTok{log}\NormalTok{(Pdata}\OperatorTok{$}\NormalTok{time)}
\KeywordTok{plot}\NormalTok{(Pdata}\OperatorTok{$}\NormalTok{logtime, Pdata}\OperatorTok{$}\NormalTok{prop, }\DataTypeTok{xlab =} \StringTok{"log(Time since last memorization in minutes)"}\NormalTok{, }\DataTypeTok{ylab =} \StringTok{"Proportion of items remembered"}\NormalTok{, }\DataTypeTok{main =} \StringTok{"Time since last memorization of a list of disconnected items vs. Proposition of items remembered"}\NormalTok{)}
\end{Highlighting}
\end{Shaded}

\includegraphics{Hwk2FinDraftSp19_files/figure-latex/unnamed-chunk-4-1.pdf}

\begin{enumerate}
\def\labelenumi{\alph{enumi})}
\setcounter{enumi}{4}
\tightlist
\item
  Run the linear regression using the transformation you suggested in c
  above and get the residual plot and qqPlot of residuals and see if
  your transformed data produces a linear regression that appears
  appropriate.
\end{enumerate}

\begin{Shaded}
\begin{Highlighting}[]
\KeywordTok{library}\NormalTok{(MASS)}
\NormalTok{Plogdata.lm <-}\StringTok{ }\KeywordTok{lm}\NormalTok{(prop}\OperatorTok{~}\NormalTok{logtime, }\DataTypeTok{data =}\NormalTok{ Pdata)}
\NormalTok{Resids=}\KeywordTok{studres}\NormalTok{(Plogdata.lm)}
\KeywordTok{plot}\NormalTok{(Plogdata.lm}\OperatorTok{$}\NormalTok{fitted.values,Resids)}
\KeywordTok{abline}\NormalTok{(}\DataTypeTok{h=}\DecValTok{0}\NormalTok{)}
\end{Highlighting}
\end{Shaded}

\includegraphics{Hwk2FinDraftSp19_files/figure-latex/unnamed-chunk-5-1.pdf}

\begin{Shaded}
\begin{Highlighting}[]
\KeywordTok{library}\NormalTok{(car)}
\KeywordTok{qqPlot}\NormalTok{(Plogdata.lm}\OperatorTok{$}\NormalTok{residuals)}
\end{Highlighting}
\end{Shaded}

\includegraphics{Hwk2FinDraftSp19_files/figure-latex/unnamed-chunk-5-2.pdf}

\begin{verbatim}
## [1]  8 13
\end{verbatim}

Both the residual plot and the qqplot look great, the data is now
homoskedastic and the y's are normally distributed around each x.

\begin{enumerate}
\def\labelenumi{\alph{enumi})}
\setcounter{enumi}{5}
\tightlist
\item
  Run the regression using your transformed data. Does the proportion
  remembered decrease with time? Test at \(\alpha = .05\), being sure to
  state hypotheses, test statistic, EXACT p-value, and conclusions.
  \(H_0\) The proportion does not decrease with time
  \(B_1\)\textgreater{}=0 \(H_A\) The proportion decreases with time
  \(B_1\)\textless{}0
\end{enumerate}

\begin{Shaded}
\begin{Highlighting}[]
\KeywordTok{summary}\NormalTok{(Plogdata.lm)}
\end{Highlighting}
\end{Shaded}

\begin{verbatim}
## 
## Call:
## lm(formula = prop ~ logtime, data = Pdata)
## 
## Residuals:
##       Min        1Q    Median        3Q       Max 
## -0.027656 -0.007620 -0.003231  0.008787  0.029406 
## 
## Coefficients:
##              Estimate Std. Error t value Pr(>|t|)    
## (Intercept)  0.843231   0.010064   83.79  < 2e-16 ***
## logtime     -0.079795   0.001713  -46.59 5.45e-14 ***
## ---
## Signif. codes:  0 '***' 0.001 '**' 0.01 '*' 0.05 '.' 0.1 ' ' 1
## 
## Residual standard error: 0.01658 on 11 degrees of freedom
## Multiple R-squared:  0.995,  Adjusted R-squared:  0.9945 
## F-statistic:  2171 on 1 and 11 DF,  p-value: 5.449e-14
\end{verbatim}

P(t-11df\textgreater{}-46.59)= 2.72e-14 As 2.72e-14 is less than our
alpha value of 0.05, we reject the null hypothesis and conclude that the
proportion does in fact decrease with time. g) State AND INTERPRET the
\(r^2\) from the regression you used. \(r^2\) for the regression is
0.995, this means that 99.5\% of the variation in the proportion of
items remembered can be explained by the variation in log(time give that
the items must be remembered for).

\begin{enumerate}
\def\labelenumi{\alph{enumi})}
\setcounter{enumi}{7}
\item
  State and interpret 1 - \(r^2\) from the regression you used. 1-
  \(r^2\) for the regression is 0.005, this means that 0.5\% of the
  variation in the proportion of items remembered cannot be explained by
  the variation in log(time give that the items must be remembered for).
\item
  Would we expect the proportion remembered from people who are asked to
  remember after 1000 minutes to exceed 27\%? (You might have to type
  ?predict in your console to figure out how to get what you need.)
\end{enumerate}

\begin{Shaded}
\begin{Highlighting}[]
\NormalTok{newdata=}\KeywordTok{data.frame}\NormalTok{(}\DataTypeTok{logtime=}\KeywordTok{log}\NormalTok{(}\DecValTok{1000}\NormalTok{))}
\KeywordTok{predict.lm}\NormalTok{(Plogdata.lm, newdata, }\DataTypeTok{interval=}\StringTok{"confidence"}\NormalTok{, }\DataTypeTok{level=}\FloatTok{0.95}\NormalTok{)}
\end{Highlighting}
\end{Shaded}

\begin{verbatim}
##         fit       lwr       upr
## 1 0.2920265 0.2800851 0.3039678
\end{verbatim}

We can be 95\% confident that on average the proportion remembered for
people asked to remember after 1000 minutes will be between 0.280 and
0.304. Thus, as this entire range exceeds 27\%, we can be at least 95\%
confident that the proportion will exceed 27\% for people given 1000
minutes. j) Noah Ohnoah is asked to remember this list after waiting 500
minutes since seeing it. What proportion will Noah remember?

\begin{Shaded}
\begin{Highlighting}[]
\NormalTok{newdata=}\KeywordTok{data.frame}\NormalTok{(}\DataTypeTok{logtime=}\KeywordTok{log}\NormalTok{(}\DecValTok{500}\NormalTok{))}
\KeywordTok{predict.lm}\NormalTok{(Plogdata.lm, newdata, }\DataTypeTok{interval=}\StringTok{"prediction"}\NormalTok{, }\DataTypeTok{level=}\FloatTok{0.95}\NormalTok{)}
\end{Highlighting}
\end{Shaded}

\begin{verbatim}
##         fit       lwr       upr
## 1 0.3473362 0.3092809 0.3853915
\end{verbatim}

The probability that Noah will remember between 30.9\% and 38.5\% of
items after 500 minutes is 95\%. \pagebreak

\section{Question 2.}\label{question-2.}

The use of insecticides is beneficial for increasing agricultural
production but is a major concern for consumers' advocates and
environmentalists. Insecticides protect crops against insect damage, but
insecticide use may be harmful to humans.

A horticultural researcher working for New York State extension in
Oneida County, in central New York, would like to investigate the
relationship between the size of apples and the concentration of a new
insecticide (ppm) retained in them. She first applies the insecticide
across their experimental orchard following guidelines set forward by
USDA. The orchard contains dozens of each apple variety commonly grown
in New York. She then randomly selects 41 trees and harvests 1 apple on
each of these sampled trees. The amount of insecticide retained by each
apple is determined in the lab. The diameter of each apple is also
measured. (Data for this problem can be found in the file
\texttt{Hwk2Q3DatSp18.xlsx})

The average apple in New York State is 6.6 cm across. State regulations
for use of this insecticide, which has been shown to be extremely
effective and relatively inexpensive, require the average sized apple to
contain less than 2.8 ppm of insecticide, on average. Do these data show
that the insecticide can be allowed for use in New York State?

\begin{enumerate}
\def\labelenumi{\alph{enumi})}
\tightlist
\item
  Formulation of the research question and choice of the appropriate
  statistical technique technique used to answer this question.
\end{enumerate}

\begin{Shaded}
\begin{Highlighting}[]
\KeywordTok{library}\NormalTok{(readxl)}
\KeywordTok{getwd}\NormalTok{()}
\end{Highlighting}
\end{Shaded}

\begin{verbatim}
## [1] "/Users/kevinklaben/Downloads"
\end{verbatim}

\begin{Shaded}
\begin{Highlighting}[]
\NormalTok{Qtwodata <-}\StringTok{ }\KeywordTok{read_excel}\NormalTok{(}\StringTok{"Hwk2Q2DatSp19.xlsx"}\NormalTok{)}
\KeywordTok{plot}\NormalTok{(Qtwodata}\OperatorTok{$}\StringTok{`}\DataTypeTok{Diameter of the Apple (X in cm)}\StringTok{`}\NormalTok{, Qtwodata}\OperatorTok{$}\StringTok{`}\DataTypeTok{Insecticide Dose Found (Y, in ppm)}\StringTok{`}\NormalTok{, }\DataTypeTok{xlab =} \StringTok{"Diameter of the Apple (X in cm)"}\NormalTok{, }\DataTypeTok{ylab =} \StringTok{"Insecticide Dose Found (Y, in ppm)"}\NormalTok{, }\DataTypeTok{main =} \StringTok{"Diameter of the Apple (X in cm) vs. Insecticide Dose Found (Y, in ppm) in New York state apples"}\NormalTok{)}
\end{Highlighting}
\end{Shaded}

\includegraphics{Hwk2FinDraftSp19_files/figure-latex/unnamed-chunk-9-1.pdf}

Research Question. Via the plot, we can see that the question should be
whether there is a linear relationship that exists between some
transformation of the Diameter of the Apple (in cm) and Insecticide Dose
Found (in ppm), so that we could use Diameter of the Apple (in cm) to
predict Insecticide Dose Found (in ppm). Statistical Technique. Fit a
linear model with some transformation of Diameter of the Apple (in cm)
being the explanatory variable and some transformation of Insecticide
Dose Found (in ppm) being the response variable. Then we use two-sided
t-test to test the significance of β1 (the slope). If the model appears
to be siginifant, we then use this model to perform a hypothesis test
with for apple diameter=6.6 with \(H_0\) = the insecticide for the
average sized apple is on average greater than or equal to 2.8 ppm.
\(H_A\) = the insecticide for the average sized apple is on average less
than 2.8 ppm.

\begin{Shaded}
\begin{Highlighting}[]
\KeywordTok{library}\NormalTok{(MASS)}
\NormalTok{Qtwodata.lm <-}\StringTok{ }\KeywordTok{lm}\NormalTok{(}\StringTok{`}\DataTypeTok{Insecticide Dose Found (Y, in ppm)}\StringTok{`}\OperatorTok{~}\StringTok{`}\DataTypeTok{Diameter of the Apple (X in cm)}\StringTok{`}\NormalTok{, }\DataTypeTok{data =}\NormalTok{ Qtwodata)}
\NormalTok{Resids=}\KeywordTok{studres}\NormalTok{(Qtwodata.lm)}
\KeywordTok{plot}\NormalTok{(Qtwodata.lm}\OperatorTok{$}\NormalTok{fitted.values,Resids)}
\KeywordTok{abline}\NormalTok{(}\DataTypeTok{h=}\DecValTok{0}\NormalTok{)}
\end{Highlighting}
\end{Shaded}

\includegraphics{Hwk2FinDraftSp19_files/figure-latex/unnamed-chunk-10-1.pdf}

\begin{Shaded}
\begin{Highlighting}[]
\KeywordTok{library}\NormalTok{(car)}
\KeywordTok{qqPlot}\NormalTok{(Qtwodata.lm}\OperatorTok{$}\NormalTok{residuals)}
\end{Highlighting}
\end{Shaded}

\includegraphics{Hwk2FinDraftSp19_files/figure-latex/unnamed-chunk-10-2.pdf}

\begin{verbatim}
## [1] 41 38
\end{verbatim}

\begin{Shaded}
\begin{Highlighting}[]
\NormalTok{Qtwodata}\OperatorTok{$}\NormalTok{loginsect<-}\KeywordTok{log}\NormalTok{(Qtwodata}\OperatorTok{$}\StringTok{`}\DataTypeTok{Insecticide Dose Found (Y, in ppm)}\StringTok{`}\NormalTok{)}
\NormalTok{Qtwodata}\OperatorTok{$}\NormalTok{logapp<-(Qtwodata}\OperatorTok{$}\StringTok{`}\DataTypeTok{Diameter of the Apple (X in cm)}\StringTok{`}\NormalTok{)}

\KeywordTok{plot}\NormalTok{(Qtwodata}\OperatorTok{$}\NormalTok{logapp, Qtwodata}\OperatorTok{$}\NormalTok{loginsect, }\DataTypeTok{xlab =} \StringTok{"Diameter of the Apple (X in cm)"}\NormalTok{, }\DataTypeTok{ylab =} \StringTok{"Insecticide Dose Found (Y, in ppm)"}\NormalTok{, }\DataTypeTok{main =} \StringTok{"Diameter of the Apple (X in cm) vs. Insecticide Dose Found (Y, in ppm) in New York state apples"}\NormalTok{)}
\end{Highlighting}
\end{Shaded}

\includegraphics{Hwk2FinDraftSp19_files/figure-latex/unnamed-chunk-10-3.pdf}

\begin{Shaded}
\begin{Highlighting}[]
\KeywordTok{library}\NormalTok{(MASS)}
\NormalTok{Qtwodataloginsect.lm <-}\StringTok{ }\KeywordTok{lm}\NormalTok{(loginsect}\OperatorTok{~}\NormalTok{logapp, }\DataTypeTok{data =}\NormalTok{ Qtwodata)}
\NormalTok{Resids=}\KeywordTok{studres}\NormalTok{(Qtwodataloginsect.lm)}
\KeywordTok{plot}\NormalTok{(Qtwodataloginsect.lm}\OperatorTok{$}\NormalTok{fitted.values,Resids)}
\KeywordTok{abline}\NormalTok{(}\DataTypeTok{h=}\DecValTok{0}\NormalTok{)}
\end{Highlighting}
\end{Shaded}

\includegraphics{Hwk2FinDraftSp19_files/figure-latex/unnamed-chunk-10-4.pdf}

\begin{Shaded}
\begin{Highlighting}[]
\KeywordTok{library}\NormalTok{(car)}
\KeywordTok{qqPlot}\NormalTok{(Qtwodataloginsect.lm}\OperatorTok{$}\NormalTok{residuals)}
\end{Highlighting}
\end{Shaded}

\includegraphics{Hwk2FinDraftSp19_files/figure-latex/unnamed-chunk-10-5.pdf}

\begin{verbatim}
## [1] 2 6
\end{verbatim}

\begin{enumerate}
\def\labelenumi{\alph{enumi})}
\setcounter{enumi}{1}
\tightlist
\item
  Notation for the random variable(s) and parameter(s) of interest;
  define these explicitly. Give the distributional assumptions for your
  random variable(s) and state all assumptions necessary for the
  statistical application you intend to use.
\end{enumerate}

\textbf{Notation.} Let \(Y_i\) be the Amount of Insecticide (in ppm) and
\(X_i\) be the log of apple diameter. Then we could write our proposed
linear models as

\begin{align}
Y_i=\beta_0+\beta_1log(X_i)+\epsilon_i,
\end{align}

where \(\epsilon_i\) is the error term for the \(i\)th observation.

\textbf{Assumptions.} We make the following assumptions for the model:

\begin{enumerate}
\item \emph{(Independence).} Observations (and hence residuals) are independent
\item \emph{(Normality).} $\epsilon_i\overset{i.i.d}{\sim}\mathrm{N}(0,\sigma^2)$
\item \emph{(Linearity).} Means of $\log(Y_i)$ are linearly related to $X_i$.
\item \emph{(Homoscedasticity).} $\epsilon_i$ have constanct variance.
\item \emph{(Negligible Outliers).} Outliers are not driving our conclusions
\end{enumerate}

\begin{enumerate}
\def\labelenumi{\alph{enumi})}
\setcounter{enumi}{2}
\tightlist
\item
  Calculations for the analysis. For hypothesis and significance tests,
  formulate the null and the alternative hypotheses, calculate the value
  of your test statistic, and then calculate your p-value. For
  confidence intervals, show and apply the appropriate formula. Use
  \(\alpha=0.10\).
\end{enumerate}

First, we test the siginificance of the linear model. Formally, we are
testing

\begin{align}
H_0: \beta_1 = 0 \text{ against } H_1: \beta_1\neq 0
\end{align}

To do this, we have

\begin{Shaded}
\begin{Highlighting}[]
\NormalTok{X =}\StringTok{ }\NormalTok{Qtwodata}\OperatorTok{$}\NormalTok{logapp}
\NormalTok{Y =}\StringTok{ }\NormalTok{Qtwodata}\OperatorTok{$}\NormalTok{loginsect}
\NormalTok{proposed.model =}\StringTok{ }\KeywordTok{lm}\NormalTok{(Y}\OperatorTok{~}\NormalTok{X)}
\KeywordTok{summary}\NormalTok{(proposed.model)}
\end{Highlighting}
\end{Shaded}

\begin{verbatim}
## 
## Call:
## lm(formula = Y ~ X)
## 
## Residuals:
##      Min       1Q   Median       3Q      Max 
## -0.57704 -0.03159 -0.01361  0.00227  0.60178 
## 
## Coefficients:
##              Estimate Std. Error t value Pr(>|t|)    
## (Intercept) -15.32433    0.32368  -47.34   <2e-16 ***
## X             2.42836    0.04993   48.63   <2e-16 ***
## ---
## Signif. codes:  0 '***' 0.001 '**' 0.01 '*' 0.05 '.' 0.1 ' ' 1
## 
## Residual standard error: 0.2034 on 39 degrees of freedom
## Multiple R-squared:  0.9838, Adjusted R-squared:  0.9834 
## F-statistic:  2365 on 1 and 39 DF,  p-value: < 2.2e-16
\end{verbatim}

\textbf{Conclusion.} From the summary, we see that the test of
\(\beta_1\) has a \(p\)-value of 2e-16, which is significantly smaller
than 0.05. Therefore, we reject the null hypothesis and conclude that
the linear relationship between expired volume and oxygen uptake is
significant .

Secondarily, we make a hypothesis test for \(X=6.6\) with alpha=0.10.
\(H_0\) = the insecticide for the average sized apple is on average
greater than or equal to 2.8 ppm. \(H_A\) = the insecticide for the
average sized apple is on average less than 2.8 ppm.

\begin{Shaded}
\begin{Highlighting}[]
\NormalTok{confinv =}\StringTok{ }\KeywordTok{predict.lm}\NormalTok{(proposed.model,}\KeywordTok{data.frame}\NormalTok{(}\DataTypeTok{X=}\FloatTok{6.6}\NormalTok{),}\DataTypeTok{se.fit=}\OtherTok{TRUE}\NormalTok{,}\DataTypeTok{interval=}\StringTok{"confidence"}\NormalTok{, }\DataTypeTok{level=}\FloatTok{0.95}\NormalTok{)}
\end{Highlighting}
\end{Shaded}

We know that the t-value will be = ((pt. Est.)-(hypothesized
value))/Se(pt est) using a confidence interval we can determine that the
pt. est. is 0.703 and the standard error=0.0326

\begin{Shaded}
\begin{Highlighting}[]
\NormalTok{tvalue=}\StringTok{ }\NormalTok{(}\FloatTok{0.703}\OperatorTok{-}\KeywordTok{log}\NormalTok{(}\FloatTok{2.8}\NormalTok{))}\OperatorTok{/}\FloatTok{0.0326}
\NormalTok{pval=}\KeywordTok{pt}\NormalTok{(}\OperatorTok{-}\FloatTok{10.019}\NormalTok{,}\DataTypeTok{df=}\DecValTok{39}\NormalTok{)}
\end{Highlighting}
\end{Shaded}

Thus, the t-value= (0.703-log(2.8))/0.0326=-10.019
p-val=pt(-10.019,df=n-1)=1.212e-12 As the p-value of
1.212e-12\textless{} 0.1, we reject the null hypothesis and conclude
that the amount of insecticide in average sized apples will on average
be less than 2.8 ppm.

\begin{enumerate}
\def\labelenumi{\alph{enumi})}
\setcounter{enumi}{3}
\tightlist
\item
  Discuss whether the assumptions stated in part B above are met
  sufficiently for the validity of the statistical inferences; use
  graphs and other tools where applicable.
\end{enumerate}

\emph{(Independence).} Each sampled apple is independent from the next
if the target population was apples at this specific farm as they are
chosen randomly from different trees.

\emph{(Normality of Residuals).} We do qqplot to check for normality.

\begin{Shaded}
\begin{Highlighting}[]
\KeywordTok{library}\NormalTok{(car)}
\CommentTok{# qqplot for checking normality}
\NormalTok{std.residual =}\StringTok{ }\KeywordTok{rstandard}\NormalTok{(proposed.model)}
\KeywordTok{qqPlot}\NormalTok{(std.residual, }\DataTypeTok{main=}\StringTok{"QQ plot with confidence intervals"}\NormalTok{)}
\end{Highlighting}
\end{Shaded}

\includegraphics{Hwk2FinDraftSp19_files/figure-latex/unnamed-chunk-14-1.pdf}

\begin{verbatim}
## [1] 2 6
\end{verbatim}

Since many points (probably more than 5\%) fall outside of the
confidence bands, the normality assumption is not totally valid here.
However, the bands are quite narrow with this large of a data set, and
the deviation around the line is mostly on the tails. Therefore, we can
trust that the robustness of the t-procedures used here against minor to
moderate deviations from normality make our conclusions valid in this
case.

\emph{(Linearity).} It can be verified from the previous scatterplot.

\emph{(Homoskedasticity).} The relatively constant variance around the
line seems apparent from the residual plot, and so this assumption
appears to be met. Take a better look with a residual plot.

\begin{Shaded}
\begin{Highlighting}[]
\CommentTok{# residuals vs fits for checking homoscedasticity}
\KeywordTok{plot}\NormalTok{(proposed.model}\OperatorTok{$}\NormalTok{fit, std.residual, }\DataTypeTok{xlab =}\StringTok{"Fits"}\NormalTok{, }\DataTypeTok{ylab=}\StringTok{"Standardized Residuals"}\NormalTok{)}
\KeywordTok{abline}\NormalTok{(}\DataTypeTok{h=}\DecValTok{0}\NormalTok{)}
\end{Highlighting}
\end{Shaded}

\includegraphics{Hwk2FinDraftSp19_files/figure-latex/unnamed-chunk-15-1.pdf}

Most of the points in the above residual plot distribute evenly above
and below the zero line, so homoscedasticity assumption is valid.

\emph{(Negligible Outliers).} There were no outliers apparent from any
of the diagnostic plots other than the tails, Let's take a look at a
Cook's Distance plot to see if any are influential.

\begin{Shaded}
\begin{Highlighting}[]
\KeywordTok{plot}\NormalTok{(proposed.model, }\DataTypeTok{which =} \DecValTok{4}\NormalTok{)}
\end{Highlighting}
\end{Shaded}

\includegraphics{Hwk2FinDraftSp19_files/figure-latex/unnamed-chunk-16-1.pdf}

With all Cook's Distances less than 0.5, no particular point seems to be
strongly influencing our inferences

\begin{enumerate}
\def\labelenumi{\alph{enumi})}
\setcounter{enumi}{4}
\item
  Discuss the sampling scheme and whether or not it is sufficient to
  meet the objective of the study. Be sure to include whether or not
  subjective inference is necessary and if so, defend whether or not you
  believe it is valid. The data was randomly sampled from a single farm
  in Oneida, NY. This is not suffient on its own to meet the objective
  of this study as the sampled population (apples at this specific
  Oneida farm) is not the same as the target population (Apples in New
  York State). Thus, subjective inference is required to validate our
  findings. However, this data may not be reliable to transfer to the
  population of new york state apples in general, as different farms and
  slightly different climate throughout the state may have some effect
  on insecticide retention. As these are simply speculated possibilities
  and not facts we have no legitimate reason to believe that the sampled
  population is not representative of the target population, however, if
  a link between climate or farm technique and insecticide retention is
  discovered, our data will be cast into doubt.
\item
  State the conclusions of the analysis. These should be practical
  conclusions from the context of the problem, but should also be backed
  up with statistical criteria (like a p-value, etc.). Include any
  considerations such as limitations of the sampling scheme, impact of
  outliers, etc., that you feel must be considered when you state your
  conclusions.
\end{enumerate}

We may conclude tentatively that for apples 6.6 cm in diameter, they
will on average contain less than 2.8 ppm insecticide with a p value of
1.212e-12. Despite the limitations, which include tails which vary by a
larger amount than the rest of the data, and the sampling scheme selects
from a sample population that may not completely be representative of
the overall population of interest, we can be fairly certain that this
conclusion is valid as the results are much stronger than required as a
p value of 1.212e-12 is very strong. Thus, despite the limitations of
the data we can be fairly confident thatthis insecticide follows the NYS
reuglations.


\end{document}
