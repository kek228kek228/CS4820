\documentclass[12pt]{article}
%\usepackage{fullpage}
\usepackage{epic}
\usepackage{eepic}
\usepackage{paralist}
\usepackage{graphicx}
\usepackage{algorithm,algorithmic}
\usepackage{tikz}
\usepackage{xcolor,colortbl}
\usepackage{wrapfig}
\usepackage{amsmath}


%%%%%%%%%%%%%%%%%%%%%%%%%%%%%%%%%%%%%%%%%%%%%%%%%%%%%%%%%%%%%%%%
% This is FULLPAGE.STY by H.Partl, Version 2 as of 15 Dec 1988.
% Document Style Option to fill the paper just like Plain TeX.

\typeout{Style Option FULLPAGE Version 2 as of 15 Dec 1988}

\topmargin 0pt
\advance \topmargin by -\headheight
\advance \topmargin by -\headsep

\textheight 8.9in

\oddsidemargin 0pt
\evensidemargin \oddsidemargin
\marginparwidth 0.5in

\textwidth 6.5in
%%%%%%%%%%%%%%%%%%%%%%%%%%%%%%%%%%%%%%%%%%%%%%%%%%%%%%%%%%%%%%%%

\pagestyle{empty}
\setlength{\oddsidemargin}{0in}
\setlength{\topmargin}{-0.8in}
\setlength{\textwidth}{6.8in}
\setlength{\textheight}{9.5in}


\def\ind{\hspace*{0.3in}}
\def\gap{0.1in}
\def\bigap{0.25in}
\newcommand{\Xomit}[1]{}
\newcommand{\Mod}[1]{\ (\mathrm{mod}\ #1)}


\begin{document}

\setlength{\parindent}{0in}
\addtolength{\parskip}{0.1cm}
\setlength{\fboxrule}{.5mm}\setlength{\fboxsep}{1.2mm}
\newlength{\boxlength}\setlength{\boxlength}{\textwidth}
\addtolength{\boxlength}{-4mm}
\begin{center}\framebox{\parbox{\boxlength}{{\bf
CS 4820, Spring 2019 \hfill Homework 8, Problem 2}\\
% TODO: fill in your own name, netID, and collaborators
Name: \\
NetID: \\
}}
\end{center}
\vspace{5mm}
\vskip \bigap
{\bf (2)} {\em (15 points)}\\
Recall that in the Knapsack Problem, one is given
a set of items numbered $1,2,\ldots,n$,
such that the $i$-th item has value $v_i \geq 0$
and weight $w_i \geq 0.$  Given a total weight
constraint $W$, the problem is to choose a
subset $S \subseteq \{1,2,\ldots,n\}$ so
as to maximize the combined value, $\sum_{i \in S} v_i$,
subject to the weight constraint $\sum_{i \in S} w_i \leq W.$
\vskip \gap
{\bf (a)}
Consider the following \emph{greedy algorithm}, GA.
\begin{enumerate}
\item Eliminate items whose weight $w_i$ is greater than $W$.
\item For each remaining $i$, 
compute the \emph{value density} $\rho_i = v_i/w_i.$
\item Sort the remaining items in order of decreasing $\rho_i$.
\item Choose the longest initial segment of this sorted list 
that does not violate the size constraint.
\end{enumerate}
Also consider the following
\emph{even more greedy algorithm}, EMGA.
\begin{enumerate}
\item Eliminate items whose weight $w_i$ is greater than $W$.
\item Sort the remaining items in order of decreasing $v_i$.
\item Choose the longest initial segment of this sorted
list that does not violate the size constraint.
\end{enumerate}
For each of these two algorithms, give a
counterexample to demonstrate 
that its approximation ratio 
is not bounded above by any constant $C$.
(Use different counterexamples for the two algorithms.)
\vskip \gap
{\bf (b)}
Now consider the following algorithm: run GA and EMGA,
look at the two solutions they produce, and pick the
one with higher total value.  Prove that this is
a 2-approximation algorithm for the Knapsack Problem,
i.e.\ it selects a set whose value is at least half
of the value of the optimal set.
\vskip \gap
{\bf (c)}
By combining part (b) with the dynamic programming
algorithm for Knapsack presented in class, show that
for every $\varepsilon>0$, there is a Knapsack algorithm
with running time $O(n^2 s / \varepsilon)$ whose
approximation ratio is at most $1+\varepsilon.$
Here, $s$ denotes the maximum number of bits in the
binary representation of any of the numbers $v_i, w_i \;
(i=1,\ldots,n)$.
[We will see an  algorithm in class with
running time $O(n^3 s / \varepsilon).$]
In your solution, it is not necessary to repeat 
the proof of correctness of the algorithm that will be presented in class.



%% Your solution goes here.


\end{document}
