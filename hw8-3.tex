\documentclass[12pt]{article}
%\usepackage{fullpage}
\usepackage{epic}
\usepackage{eepic}
\usepackage{paralist}
\usepackage{graphicx}
\usepackage{algorithm,algorithmic}
\usepackage{tikz}
\usepackage{xcolor,colortbl}
\usepackage{wrapfig}
\usepackage{amsmath}


%%%%%%%%%%%%%%%%%%%%%%%%%%%%%%%%%%%%%%%%%%%%%%%%%%%%%%%%%%%%%%%%
% This is FULLPAGE.STY by H.Partl, Version 2 as of 15 Dec 1988.
% Document Style Option to fill the paper just like Plain TeX.

\typeout{Style Option FULLPAGE Version 2 as of 15 Dec 1988}

\topmargin 0pt
\advance \topmargin by -\headheight
\advance \topmargin by -\headsep

\textheight 8.9in

\oddsidemargin 0pt
\evensidemargin \oddsidemargin
\marginparwidth 0.5in

\textwidth 6.5in
%%%%%%%%%%%%%%%%%%%%%%%%%%%%%%%%%%%%%%%%%%%%%%%%%%%%%%%%%%%%%%%%

\pagestyle{empty}
\setlength{\oddsidemargin}{0in}
\setlength{\topmargin}{-0.8in}
\setlength{\textwidth}{6.8in}
\setlength{\textheight}{9.5in}


\def\ind{\hspace*{0.3in}}
\def\gap{0.1in}
\def\bigap{0.25in}
\newcommand{\Xomit}[1]{}
\newcommand{\Mod}[1]{\ (\mathrm{mod}\ #1)}


\begin{document}

\setlength{\parindent}{0in}
\addtolength{\parskip}{0.1cm}
\setlength{\fboxrule}{.5mm}\setlength{\fboxsep}{1.2mm}
\newlength{\boxlength}\setlength{\boxlength}{\textwidth}
\addtolength{\boxlength}{-4mm}
\begin{center}\framebox{\parbox{\boxlength}{{\bf
CS 4820, Spring 2019 \hfill Homework 8, Problem 3}\\
% TODO: fill in your own name, netID, and collaborators
Name: \\
NetID: \\
}}
\end{center}
\vspace{5mm}
\vskip \bigap
{\bf (3)} {\em (10 points)} The transactions in a blockchain ledger can be modeled as a directed acyclic graph $G=(V,E)$ whose vertex set is partitioned into subsets $V_1,V_2,\ldots,V_p$, where $V_i$ represents the set of transactions pertaining to user $i$, and an edge $(u,v)$ can be interpreted as meaning that transaction $u$ is a predecessor of transaction $v$. The graph $G$ and its partition $V_1,\ldots,V_p$ are assumed to satisfy the following property:
\begin{itemize}[($\ast$)]
  \item {\em For $i=1,\ldots,p$, $V_i$ contains a node $r_i$ that has
    no incoming edges in $G$. For every other $v \in V_i$ there is at least one
    $u \in V_i$ such that $(u,v) \in E$.}
\end{itemize}
A set of transactions, $S$, is called {\em compatible} if it satisfies the following two properties.
\begin{enumerate}
  \item For all $(u,v) \in E$, if $v \in S$ then $u \in S$.
  \item For all $i = 1,2,\ldots,p$, if $V_i$ contains three distinct nodes $u,v,w$ such that
    $u$ has edges to both $v$ and $w$ in $G$, then $v$ and $w$ cannot both belong to $S$.
\end{enumerate}
The first constraint can be interpreted as stating that a transaction cannot be accepted
unless all of its predecessors are accepted. The second constraint prevents each user $i$
from ``double-spending.''

Consider the decision problem COMPAT defined as follows. An input instance consists of a directed acyclic
graph $G=(V,E)$, a partition of $V$ into subsets $V_1,\ldots, V_p$ satisfying property ($\ast$), and a
positive integer $k \le |V|$. It is a `Yes' instance of COMPAT if and only if there exists a compatible
set of at least $k$ transactions. Prove that COMPAT is NP-complete.


%% Your solution goes here.


\end{document}
